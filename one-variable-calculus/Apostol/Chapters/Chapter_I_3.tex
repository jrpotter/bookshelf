\documentclass{article}
\usepackage[shortlabels]{enumitem}

\usepackage{amsfonts, amsthm}
\usepackage{hyperref}

\newtheorem{theorem}{Theorem}
\newtheorem{xtheoreminner}{Theorem}
\newenvironment{xtheorem}[1]{%
  \renewcommand\thextheoreminner{#1}%
  \xtheoreminner
}{\endxtheoreminner}

\hypersetup{colorlinks=true, urlcolor=blue}

\newcommand{\ceil}[1]{\left\lceil#1\right\rceil}
\newcommand{\floor}[1]{\left\lfloor#1\right\rfloor}
% The first argument refers to a relative path upward from a current file to
% the root of the workspace (i.e. where this `preamble.tex` file is located).
\newcommand{\lean}[4]{\href{#1/#2.html\##3}{#4}}


\begin{document}

\begin{xtheorem}{I.27}

  Every nonempty set $S$ that is bounded below has a greatest lower bound; that
  is, there is a real number $L$ such that $L = \inf{S}$.

\end{xtheorem}

\begin{proof}

  \href{Chapter_I_3.lean}{Apostol.Chapter_I_3.Real.exists_isGLB}

\end{proof}

\begin{xtheorem}{I.29}

  For every real $x$ there exists a positive integer $n$ such that $n > x$.

\end{xtheorem}

\begin{proof}

  \href{Chapter_I_3.lean}{Apostol.Chapter_I_3.Real.exists_pnat_geq_self}

\end{proof}

\begin{xtheorem}{I.30}[Archimedean Property of the Reals]

  If $x > 0$ and if $y$ is an arbitrary real number, there exists a positive
  integer $n$ such that $nx > y$.

\end{xtheorem}

\begin{proof}

  \href{Chapter_I_3.lean}{Apostol.Chapter_I_3.Real.exists_pnat_mul_self_geq_of_pos}

\end{proof}

\begin{xtheorem}{I.31}

  If three real numbers $a$, $x$, and $y$ satisfy the inequalities
  $$a \leq x \leq a + \frac{y}{n}$$
  for every integer $n \geq 1$, then $x = a$.

\end{xtheorem}

\begin{proof}

  \href{Chapter_I_3.lean}{Apostol.Chapter_I_3.Real.forall_pnat_leq_self_leq_frac_imp_eq}

\end{proof}

\begin{xtheorem}{I.32}

  Let $h$ be a given positive number and let $S$ be a set of real numbers.
  \begin{enumerate}[(a)]
    \item If $S$ has a supremum, then for some $x$ in $S$ we have
      $$x > \sup{S} - h.$$
    \item If $S$ has an infimum, then for some $x$ in $S$ we have
      $$x < \inf{S} + h.$$
  \end{enumerate}

\end{xtheorem}

\begin{proof}

  \  % Force space prior to *Proof.*

  \begin{enumerate}[(a)]
    \item \href{Chapter_I_3.lean}{Apostol.Chapter_I_3.Real.sup_imp_exists_gt_sup_sub_delta}
    \item \href{Chapter_I_3.lean}{Apostol.Chapter_I_3.Real.inf_imp_exists_lt_inf_add_delta}
  \end{enumerate}

\end{proof}

\begin{xtheorem}{I.33}[Additive Property]

  Given nonempty subsets $A$ and $B$ of $\mathbb{R}$, let $C$ denote the set
  $$C = \{a + b : a \in A, b \in B\}.$$

  \begin{enumerate}[(a)]
    \item If each of $A$ and $B$ has a supremum, then $C$ has a supremum, and
      $$\sup{C} = \sup{A} + \sup{B}.$$
    \item If each of $A$ and $B$ has an infimum, then $C$ has an infimum, and
      $$\inf{C} = \inf{A} + \inf{B}.$$
  \end{enumerate}

\end{xtheorem}

\begin{proof}

  \  % Force space prior to *Proof.*

  \begin{enumerate}[(a)]
    \item \href{Chapter_I_3.lean}{Apostol.Chapter_I_3.Real.sup_minkowski_sum_eq_sup_add_sup}
    \item \href{Chapter_I_3.lean}{Apostol.Chapter_I_3.Real.inf_minkowski_sum_eq_inf_add_inf}
  \end{enumerate}

\end{proof}

\begin{xtheorem}{I.34}

  Given two nonempty subsets $S$ and $T$ of $\mathbb{R}$ such that
  $$s \leq t$$
  for every $s$ in $S$ and every $t$ in $T$. Then $S$ has a supremum, and $T$
  has an infimum, and they satisfy the inequality
  $$\sup{S} \leq \inf{T}.$$

\end{xtheorem}

\begin{proof}

  \href{Chapter_I_3.lean}{Apostol.Chapter_I_3.Real.forall_mem_le_forall_mem_imp_sup_le_inf}

\end{proof}

\end{document}
