\documentclass{article}

\usepackage{amsfonts, amsthm}
\usepackage{hyperref}

\newtheorem{theorem}{Theorem}
\newtheorem{xtheoreminner}{Theorem}
\newenvironment{xtheorem}[1]{%
  \renewcommand\thextheoreminner{#1}%
  \xtheoreminner
}{\endxtheoreminner}

\hypersetup{colorlinks=true, urlcolor=blue}

\newcommand{\ceil}[1]{\left\lceil#1\right\rceil}
\newcommand{\floor}[1]{\left\lfloor#1\right\rfloor}
% The first argument refers to a relative path upward from a current file to
% the root of the workspace (i.e. where this `preamble.tex` file is located).
\newcommand{\lean}[4]{\href{#1/#2.html\##3}{#4}}


\begin{document}

\begin{xtheorem}{I.27}

Every nonempty set $S$ that is bounded below has a greatest lower bound;
that is, there is a real number $L$ such that $L = \inf{S}$.

\end{xtheorem}

\begin{proof}

\href{Chapter_I_3.lean}{Apostol.Chapter_I_3.Real.exists_isGLB}

\end{proof}

\begin{xtheorem}{I.29}

For every real $x$ there exists a positive integer $n$ such that $n > x$.

\end{xtheorem}

\begin{proof}

\href{Chapter_I_3.lean}{Apostol.Chapter_I_3.Real.exists_pnat_geq_self}

\end{proof}

\begin{xtheorem}{I.30}[Archimedean Property of the Reals]

If $x > 0$ and if $y$ is an arbitrary real number, there exists a positive integer $n$ such that $nx > y$.

\end{xtheorem}

\begin{proof}

\href{Chapter_I_3.lean}{Apostol.Chapter_I_3.Real.exists_pnat_mul_self_geq_of_pos}

\end{proof}

\begin{xtheorem}{I.31}

If three real numbers $a$, $x$, and $y$ satisfy the inequalities
$$a \leq x \leq a + \frac{y}{n}$$
for every integer $n \geq 1$, then $x = a$.

\end{xtheorem}

\begin{proof}

\href{Chapter_I_3.lean}{Apostol.Chapter_I_3.Real.forall_pnat_leq_self_leq_frac_imp_eq}

\end{proof}

\end{document}
