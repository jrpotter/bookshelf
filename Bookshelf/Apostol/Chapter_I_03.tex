\documentclass{article}

\usepackage{amsfonts, amsmath, amssymb, amsthm}
\usepackage{bigfoot}
\usepackage{comment}
\usepackage[shortlabels]{enumitem}
\usepackage{etoolbox}
\usepackage{environ}
\usepackage{fontawesome5}
\usepackage{mathabx, mathrsfs}
\usepackage{soul}
\usepackage{stmaryrd}
% Must load `xcolor` before `tcolorbox` and `tikz`.
\usepackage[dvipsnames]{xcolor}
\usepackage{tcolorbox}
\usepackage{tikz}
% `hyperref` comes after `xr-hyper`.
\usepackage{xr-hyper}
\usepackage{hyperref}

% Open "private" namespace.
\makeatletter

% ========================================
% General
% ========================================

\newcommand{\header}[2]{\title{#1}\author{#2}\date{}\maketitle}

% ========================================
% Dividers
% ========================================

\newcommand\@linespace{\vspace{10pt}}
\newcommand\linedivider{\@linespace\hrule\@linespace}
\WithSuffix\newcommand\linedivider*{\@linespace\hrule}
\newcommand\suitdivider{$$\spadesuit\;\spadesuit\;\spadesuit$$}

% ========================================
% Linking
% ========================================

\hypersetup{colorlinks=true, linkcolor=blue, urlcolor=blue}
\newcommand{\textref}[1]{\text{\nameref{#1}}}
\newcommand{\hyperlabel}[1]{%
  \label{#1}%
  \hypertarget{#1}{}}

% Links to theorems/statements/etc. that can be found in Mathlib4's index.
\newcommand\@leanlink[3]{%
  \textcolor{BlueViolet}{\raisebox{-4.5pt}{%
    \tikz{\draw (0, 0) node[yscale=-1,xscale=1] {\faFont};}}{-\;}}%
  \href{https://leanprover-community.github.io/mathlib4_docs/#1.html\##2}%
  {\color{BlueViolet}{#3}}}

\newcommand\lean[2]{%
  \noindent\@leanlink{#1}{#2}{#2}}
\WithSuffix\newcommand\lean*[2]{%
  \vspace{6pt}\lean{#1}{#2}}

\newcommand\leanp[3]{%
  \noindent\@leanlink{#1}{#2}{#3}}
\WithSuffix\newcommand\leanp*[3]{%
  \vspace{6pt}\leanp{#1}{#2}{#3}}

% Links to theorems/statements/etc. found in custom index.
\newcommand\@codelink[4]{%
  \textcolor{MidnightBlue}{\raisebox{-4.5pt}{%
    \tikz{\draw (0, 0) node[xshift=8pt] {\faCodeBranch};}}{-\;}}%
  \href{#1/#2.html\##3}%
  {\color{MidnightBlue}{#4}}}

\newcommand\coderef[3]{%
  \@codelink{#1}{#2}{#3}{#3}}
\newcommand\codepref[4]{%
  \@codelink{#1}{#2}{#3}{#4}}

% Macro to build our `code` commands relative to a given directory. For
% instance, we expect to have invocation `\makecode{..}` if the TeX file exists
% one directory deep from the root of our project..
\newcommand\makecode[1]{%
  \newcommand\code[2]{%
    \noindent\coderef{#1}{##1}{##2}}
  \WithSuffix\newcommand\code*[2]{%
    \vspace{6pt}\noindent\coderef{#1}{##1}{##2}}

  \newcommand\codep[3]{%
    \noindent\codepref{#1}{##1}{##2}{##3}}
  \WithSuffix\newcommand\codep*[3]{%
    \vspace{6pt}\noindent\codepref{#1}{##1}{##2}{##3}}
}

% ========================================
% Admonitions
% ========================================

\NewEnviron{note}{%
  \begin{tcolorbox}[%
      sharp corners,
      fonttitle=\sffamily\bfseries,
      toptitle=2pt,
      bottomtitle=2pt,
      coltitle=black!80!white,
      colback=yellow!30,
      colframe=yellow!80!black,
      title=Note]
    \BODY
  \end{tcolorbox}}

% ========================================
% Statements
% ========================================

\newcommand\@statement[1]{%
  \linedivider*\paragraph{\normalfont\normalsize\textit{#1.}}}
\newenvironment{answer}{\@statement{Answer}}{\hfill$\square$}
\renewenvironment{proof}{\@statement{Proof}}{\hfill$\square$}

\newtheorem{corollaryinner}{Corollary}
\newenvironment{corollary}[1][]{%
  \ifstrempty{#1}
    {\corollaryinner}
    {\renewcommand\thecorollaryinner{#1}\corollaryinner}
}{\endcorollaryinner}

\newtheorem{lemmainner}{Lemma}
\newenvironment{lemma}[1][]{%
  \ifstrempty{#1}
    {\lemmainner}
    {\renewcommand\thelemmainner{#1}\lemmainner}
}{\endlemmainner}

\newtheorem{theoreminner}{Theorem}
\newenvironment{theorem}[1][]{%
  \ifstrempty{#1}
    {\theoreminner}
    {\renewcommand\thetheoreminner{#1}\theoreminner}
}{\endtheoreminner}

% ========================================
% Status
% ========================================

\DeclareRobustCommand{\defined}[1]{%
  \texorpdfstring{\color{darkgray}\faParagraph\ #1}{#1}}
\DeclareRobustCommand{\verified}[1]{%
  \texorpdfstring{\color{teal}\faCheckCircle\ #1}{#1}}
\DeclareRobustCommand{\unverified}[1]{%
  \texorpdfstring{\color{olive}\faCheckCircle[regular]\ #1}{#1}}
\DeclareRobustCommand{\pending}[1]{%
  \texorpdfstring{\color{Fuchsia}\faPencil*\ #1}{#1}}
\DeclareRobustCommand{\sorry}[1]{%
  \texorpdfstring{\color{Maroon}\faExclamationCircle\ #1}{#1}}

% ========================================
% Math
% ========================================

\newcommand{\abs}[1]{\left|#1\right|}
\newcommand{\ceil}[1]{\left\lceil#1\right\rceil}
\newcommand{\dom}[1]{\textop{dom}{#1}}
\newcommand{\fld}[1]{\textop{fld}{#1}}
\newcommand{\floor}[1]{\left\lfloor#1\right\rfloor}
\newcommand{\icc}[2]{\left[#1, #2\right]}
\newcommand{\ico}[2]{\left[#1, #2\right)}
\newcommand{\img}[2]{#1\!\left\llbracket#2\right\rrbracket}
\newcommand{\ioc}[2]{\left(#1, #2\right]}
\newcommand{\ioo}[2]{\left(#1, #2\right)}
\newcommand{\powerset}[1]{\mathscr{P}#1}
\newcommand{\ran}[1]{\textop{ran}{#1}}
\newcommand{\textop}[1]{\mathop{\text{#1}}}
\newcommand{\ubar}[1]{\text{\b{$#1$}}}

\let\oldemptyset\emptyset
\let\emptyset\varnothing

% Close off "private" namespace.
\makeatother


\newcommand{\lean}[1]{\leanref
  {./Chapter\_I\_03.html\#Apostol.Chapter\_I\_03.#1}
  {Apostol.Chapter\_I\_03.#1}}

\begin{document}

\header{A Set of Axioms for the Real-Number System}{Tom M. Apostol}

\section*{\verified{Lemma 1}}%
\hyperlabel{sec:lemma-1}%

Nonempty set $S$ has supremum $L$ if and only if set $-S$ has infimum $-L$.

\begin{proof}

  \lean{is\_lub\_neg\_set\_iff\_is\_glb\_set\_neg}

  \divider

  Suppose $L = \sup{S}$ and fix $x \in S$.
  By definition of the supremum, $x \leq L$ and $L$ is the smallest value
    satisfying this inequality.
  Negating both sides of the inequality yields $-x \geq -L$.
  Furthermore, $-L$ must be the largest value satisfying this inequality.
  Therefore $-L = \inf{-S}$.

\end{proof}

\section*{\verified{Theorem I.27}}%
\hyperlabel{sec:theorem-i.27}%

Every nonempty set $S$ that is bounded below has a greatest lower bound; that
  is, there is a real number $L$ such that $L = \inf{S}$.

\begin{proof}

  \lean{exists\_isGLB}

  \divider

  Let $S$ be a nonempty set bounded below by $x$.
  Then $-S$ is nonempty and bounded above by $x$.
  By the completeness axiom, there exists a supremum $L$ of $-S$.
  By \nameref{sec:lemma-1}, $L$ is a supremum of $-S$ if and only if $-L$ is an
    infimum of $S$.

\end{proof}

\section*{\verified{Theorem I.29}}%
\hyperlabel{sec:theorem-i.29}

For every real $x$ there exists a positive integer $n$ such that $n > x$.

\begin{proof}

  \lean{exists\_pnat\_geq\_self}

  \divider

  Let $n = \abs{\ceil{x}} + 1$.
  It is trivial to see $n$ is a positive integer satisfying $n \geq 1$.
  Thus all that remains to be shown is that $n > x$.
  If $x$ is nonpositive, $n > x$ immediately follows from $n \geq 1$.
  If $x$ is positive,
    $$x = \abs{x} \leq \abs{\ceil{x}} < \abs{\ceil{x}} + 1 = n.$$

\end{proof}

\section*{\verified{Theorem I.30}}%
\hyperlabel{sec:theorem-i.30}%

If $x > 0$ and if $y$ is an arbitrary real number, there exists a positive
  integer $n$ such that $nx > y$.

\note{This is known as the "Archimedean Property of the Reals."}

\begin{proof}

  \lean{exists\_pnat\_mul\_self\_geq\_of\_pos}

  \divider

  Let $x > 0$ and $y$ be an arbitrary real number.
  By \nameref{sec:theorem-i.29}, there exists a positive integer $n$ such that
    $n > y / x$.
  Multiplying both sides of the inequality yields $nx > y$ as expected.

\end{proof}

\section*{\verified{Theorem I.31}}%
\hyperlabel{sec:theorem-i.31}%

If three real numbers $a$, $x$, and $y$ satisfy the inequalities
  $$a \leq x \leq a + \frac{y}{n}$$ for every integer $n \geq 1$, then $x = a$.

\begin{proof}

  \lean{forall\_pnat\_leq\_self\_leq\_frac\_imp\_eq}

  \divider

  By the trichotomy of the reals, there are three cases to consider:

  \paragraph{Case 1}%

    Suppose $x = a$.
    Then we are immediately finished.

  \paragraph{Case 2}%

    Suppose $x < a$.
    But by hypothesis, $a \leq x$.
    Thus $a < a$, a contradiction.

  \paragraph{Case 3}%

    Suppose $x > a$.
    Then there exists some $c > 0$ such that $a + c = x$.
    By \nameref{sec:theorem-i.30}, there exists an integer $n > 0$ such that
      $nc > y$.
    Rearranging terms, we see $y / n < c$.
    Therefore $a + y / n < a + c = x$.
    But by hypothesis, $x \leq a + y / n$.
    Thus $a + y / n < a + y / n$, a contradiction.

  \paragraph{Conclusion}%

    Since these cases are exhaustive and both case 2 and 3 lead to
      contradictions, $x = a$ is the only possibility.

\end{proof}

\section*{\verified{Lemma 2}}%
\hyperlabel{sec:lemma-2}%

If three real numbers $a$, $x$, and $y$ satisfy the inequalities
  $$a - y / n \leq x \leq a$$ for every integer $n \geq 1$, then $x = a$.

\begin{proof}

  \lean{forall\_pnat\_frac\_leq\_self\_leq\_imp\_eq}

  \divider

  By the trichotomy of the reals, there are three cases to consider:

  \paragraph{Case 1}%

    Suppose $x = a$.
    Then we are immediately finished.

  \paragraph{Case 2}%

    Suppose $x < a$.
    Then there exists some $c > 0$ such that $x = a - c$.
    By \nameref{sec:theorem-i.30}, there exists an integer $n > 0$ such that
      $nc > y$.
    Rearranging terms, we see that $y / n < c$.
    Therefore $a - y / n > a - c = x$.
    But by hypothesis, $x \geq a - y / n$.
    Thus $a - y / n < a - y / n$, a contradiction.

  \paragraph{Case 3}%

    Suppose $x > a$.
    But by hypothesis $x \leq a$.
    Thus $a < a$, a contradiction.

  \paragraph{Conclusion}%

    Since these cases are exhaustive and both case 2 and 3 lead to
      contradictions, $x = a$ is the only possibility.

\end{proof}

\section*{Theorem I.32}%
\hyperlabel{sec:theorem-i.32}%

Let $h$ be a given positive number and let $S$ be a set of real numbers.

\subsection*{\verified{Theorem I.32a}}%
\hyperlabel{sub:theorem-i.32a}%

If $S$ has a supremum, then for some $x$ in $S$ we have $x > \sup{S} - h$.

\begin{proof}

  \lean{sup\_imp\_exists\_gt\_sup\_sub\_delta}

  \divider

  By definition of a supremum, $\sup{S}$ is the least upper bound of $S$.
  For the sake of contradiction, suppose for all $x \in S$,
    $x \leq \sup{S} - h$.
  This immediately implies $\sup{S} - h$ is an upper bound of $S$.
  But $\sup{S} - h < \sup{S}$, contradicting $\sup{S}$ being the \textit{least}
    upper bound.
  Therefore our original hypothesis was wrong.
  That is, there exists some $x \in S$ such that $x > \sup{S} - h$.

\end{proof}

\subsection*{\verified{Theorem I.32b}}%
\hyperlabel{sub:theorem-i.32b}%

If $S$ has an infimum, then for some $x$ in $S$ we have $x < \inf{S} + h$.

\begin{proof}

  \lean{inf\_imp\_exists\_lt\_inf\_add\_delta}

  \divider

  By definition of an infimum, $\inf{S}$ is the greatest lower bound of $S$.
  For the sake of contradiction, suppose for all $x \in S$,
    $x \geq \inf{S} + h$.
  This immediately implies $\inf{S} + h$ is a lower bound of $S$.
  But $\inf{S} + h > \inf{S}$, contradicting $\inf{S}$ being the
    \textit{greatest} lower bound.
  Therefore our original hypothesis was wrong.
  That is, there exists some $x \in S$ such that $x < \inf{S} + h$.

\end{proof}

\section*{Theorem I.33}%
\hyperlabel{sec:theorem-i.33}%

Given nonempty subsets $A$ and $B$ of $\mathbb{R}$, let $C$ denote the set
  $$C = \{a + b : a \in A, b \in B\}.$$

\note{This is known as the "Additive Property."}

\subsection*{\verified{Theorem I.33a}}%
\hyperlabel{sub:theorem-i.33a}%

If each of $A$ and $B$ has a supremum, then $C$ has a supremum, and
  $$\sup{C} = \sup{A} + \sup{B}.$$

\begin{proof}

  \lean{sup\_minkowski\_sum\_eq\_sup\_add\_sup}

  \divider

  We prove (i) $\sup{A} + \sup{B}$ is an upper bound of $C$ and (ii)
    $\sup{A} + \sup{B}$ is the \textit{least} upper bound of $C$.

  \paragraph{(i)}%
  \hyperlabel{par:theorem-i.33a-i}%

    Let $x \in C$.
    By definition of $C$, there exist elements $a' \in A$ and $b' \in B$ such
      that $x = a' + b'$.
    By definition of a supremum, $a' \leq \sup{A}$.
    Likewise, $b' \leq \sup{B}$.
    Therefore $a' + b' \leq \sup{A} + \sup{B}$.
    Since $x = a' + b'$ was arbitrarily chosen, it follows $\sup{A} + \sup{B}$
      is an upper bound of $C$.

  \paragraph{(ii)}%

    Since $A$ and $B$ have supremums, $C$ is nonempty.
    By \nameref{par:theorem-i.33a-i}, $C$ is bounded above.
    Therefore the completeness axiom tells us $C$ has a supremum.
    Let $n > 0$ be an integer.
    We now prove that
      \begin{equation}
        \label{par:theorem-i.33a-ii-eq1}
        \sup{C} \leq \sup{A} + \sup{B} \leq \sup{C} + 1 / n.
      \end{equation}

    \subparagraph{Left-Hand Side}%

      First consider the left-hand side of \eqref{par:theorem-i.33a-ii-eq1}.
      By \nameref{par:theorem-i.33a-i}, $\sup{A} + \sup{B}$ is an upper bound of
        $C$.
      Since $\sup{C}$ is the \textit{least} upper bound of $C$, it follows
        $\sup{C} \leq \sup{A} + \sup{B}$.

    \subparagraph{Right-Hand Side}%

      Next consider the right-hand side of \eqref{par:theorem-i.33a-ii-eq1}.
      By \nameref{sub:theorem-i.32a}, there exists some $a' \in A$ such that
        $\sup{A} < a' + 1 / (2n)$.
      Likewise, there exists some $b' \in B$ such that
        $\sup{B} < b' + 1 / (2n)$.
      Adding these two inequalities together shows
        \begin{align*}
          \sup{A} + \sup{B}
            & < a' + b' + 1 / n \\
            & \leq \sup{C} + 1 / n.
        \end{align*}

    \subparagraph{Conclusion}%

      Applying \nameref{sec:theorem-i.31} to \eqref{par:theorem-i.33a-ii-eq1}
        proves $\sup{C} = \sup{A} + \sup{B}$ as expected.

\end{proof}

\subsection*{\verified{Theorem I.33b}}%
\hyperlabel{sub:theorem-i.33b}%

If each of $A$ and $B$ has an infimum, then $C$ has an infimum, and
  $$\inf{C} = \inf{A} + \inf{B}.$$

\begin{proof}

  \lean{inf\_minkowski\_sum\_eq\_inf\_add\_inf}

  \divider

  We prove (i) $\inf{A} + \inf{B}$ is a lower bound of $C$ and (ii)
    $\inf{A} + \inf{B}$ is the \textit{greatest} lower bound of $C$.

  \paragraph{(i)}%
  \hyperlabel{par:theorem-i.33b-i}%

    Let $x \in C$.
    By definition of $C$, there exist elements $a' \in A$ and $b' \in B$ such
      that $x = a' + b'$.
    By definition of an infimum, $a' \geq \inf{A}$.
    Likewise, $b' \geq \inf{B}$.
    Therefore $a' + b' \geq \inf{A} + \inf{B}$.
    Since $x = a' + b'$ was arbitrarily chosen, it follows $\inf{A} + \inf{B}$
      is a lower bound of $C$.

  \paragraph{(ii)}%

    Since $A$ and $B$ have infimums, $C$ is nonempty.
    By \nameref{par:theorem-i.33b-i}, $C$ is bounded below.
    Therefore \nameref{sec:theorem-i.27} tells us $C$ has an infimum.
    Let $n > 0$ be an integer.
    We now prove that
      \begin{equation}
        \label{par:theorem-i.33b-ii-eq1}
        \inf{C} - 1 / n \leq \inf{A} + \inf{B} \leq \inf{C}.
      \end{equation}

    \subparagraph{Right-Hand Side}%

      First consider the right-hand side of \eqref{par:theorem-i.33b-ii-eq1}.
      By \nameref{par:theorem-i.33b-i}, $\inf{A} + \inf{B}$ is a lower bound of
        $C$.
      Since $\inf{C}$ is the \textit{greatest} upper bound of $C$, it follows
        $\inf{C} \geq \inf{A} + \inf{B}$.

    \subparagraph{Left-Hand Side}%

      Next consider the left-hand side of \eqref{par:theorem-i.33b-ii-eq1}.
      By \nameref{sub:theorem-i.32b}, there exists some $a' \in A$ such that
        $\inf{A} > a' - 1 / (2n)$.
      Likewise, there exists some $b' \in B$ such that
        $\inf{B} > b' - 1 / (2n)$.
      Adding these two inequalities together shows
        \begin{align*}
          \inf{A} + \inf{B}
            & > a' + b' - 1 / n \\
            & \geq \inf{C} - 1 / n.
        \end{align*}

    \subparagraph{Conclusion}%

      Applying \nameref{sec:lemma-2} to \eqref{par:theorem-i.33b-ii-eq1}
        proves $\inf{C} = \inf{A} + \inf{B}$ as expected.

\end{proof}

\section*{\verified{Theorem I.34}}%
\hyperlabel{sec:theorem-i.34}%

Given two nonempty subsets $S$ and $T$ of $\mathbb{R}$ such that $$s \leq t$$
  for every $s$ in $S$ and every $t$ in $T$. Then $S$ has a supremum, and $T$
  has an infimum, and they satisfy the inequality $$\sup{S} \leq \inf{T}.$$

\begin{proof}

  \lean{forall\_mem\_le\_forall\_mem\_imp\_sup\_le\_inf}

  \divider

  By hypothesis, $S$ and $T$ are nonempty sets.
  Let $s \in S$ and $t \in T$.
  Then $t$ is an upper bound of $S$ and $s$ is a lower bound of $T$.
  By the completeness axiom, $S$ has a supremum.
  By \nameref{sec:theorem-i.27}, $T$ has an infimum.
  All that remains is showing $\sup{S} \leq \inf{T}$.

  For the sake of contradiction, suppose $\sup{S} > \inf{T}$.
  Then there exists some $c > 0$ such that $\sup{S} = \inf{T} + c$.
  Therefore $\inf{T} < \sup{S} - c / 2$.
  By \nameref{sub:theorem-i.32a}, there exists some $x \in S$ such that
    $\sup{S} - c / 2 < x$.
  Thus $$\inf{T} < \sup{S} - c / 2 < x.$$
  But by hypothesis, $x \in S$ is a lower bound of $T$ meaning $x \leq \inf{T}$.
  Therefore $x < x$, a contradiction.
  Out original assumption is incorrect; that is, $\sup{S} \leq \inf{T}$.

\end{proof}

\end{document}
