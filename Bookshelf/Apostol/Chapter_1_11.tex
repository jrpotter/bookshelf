\documentclass{article}
\usepackage{amsmath}
\usepackage[shortlabels]{enumitem}

\usepackage{amsfonts, amsthm}
\usepackage{hyperref}

\newtheorem{theorem}{Theorem}
\newtheorem{xtheoreminner}{Theorem}
\newenvironment{xtheorem}[1]{%
  \renewcommand\thextheoreminner{#1}%
  \xtheoreminner
}{\endxtheoreminner}

\hypersetup{colorlinks=true, urlcolor=blue}

\newcommand{\ceil}[1]{\left\lceil#1\right\rceil}
\newcommand{\floor}[1]{\left\lfloor#1\right\rfloor}
% The first argument refers to a relative path upward from a current file to
% the root of the workspace (i.e. where this `preamble.tex` file is located).
\newcommand{\lean}[4]{\href{#1/#2.html\##3}{#4}}


\newcommand{\link}[1]{\lean{../..}
  {Bookshelf/Apostol/Chapter\_1\_11}
  {Apostol.Chapter\_1\_11.#1}
  {Chapter\_1\_11.#1}
}

\begin{document}

\section{Exercise 4}%
\label{sec:exercise-4}

Prove that the greatest-integer function has the properties indicated:

\subsection{Exercise 4a}%
\label{sub:exercise-4a}

$\floor{x + n} = \floor{x} + n$ for every integer $n$.

\begin{proof}

  \link{exercise\_4a}

\end{proof}

\subsection{Exercise 4b}%
\label{sub:exercise-4b}

$\floor{-x} =
  \begin{cases}
    -\floor{x} & \text{if } x \text{ is an integer}, \\
    -\floor{x} - 1 & \text{otherwise}.
  \end{cases}$

\begin{proof}

  \  % Force space prior to *Proof.*

  \begin{enumerate}[(a)]
    \item \link{exercise\_4b\_1}
    \item \link{exercise\_4b\_2}
  \end{enumerate}

\end{proof}

\subsection{Exercise 4c}%
\label{sub:exercise-4c}

$\floor{x + y} = \floor{x} + \floor{y}$ or $\floor{x} + \floor{y} + 1$.

\begin{proof}

  \link{exercise\_4c}

\end{proof}

\subsection{Exercise 4d}%
\label{sub:exercise-4d}

$\floor{2x} = \floor{x} + \floor{x + \frac{1}{2}}.$

\begin{proof}

  \link{exercise\_4d}

\end{proof}

\subsection{Exercise 4e}%
\label{sub:exercise-4e}

$\floor{3x} = \floor{x} + \floor{x + \frac{1}{3}} + \floor{x + \frac{2}{3}}.$

\begin{proof}

  \link{exercise\_4e}

\end{proof}

\section{Exercise 5}%
\label{sec:exercise-5}

The formulas in Exercises 4(d) and 4(e) suggest a generalization for
  $\floor{nx}$.
State and prove such a generalization.

\begin{proof}

  \link{exercise\_5}

\end{proof}

\section{Exercise 6}%
\label{sec:exercise-6}

Recall that a lattice point $(x, y)$ in the plane is one whose coordinates are
  integers.
Let $f$ be a nonnegative function whose domain is the interval $[a, b]$, where
  $a$ and $b$ are integers, $a < b$.
Let $S$ denote the set of points $(x, y)$ satisfying $a \leq x \leq b$,
  $0 < y \leq f(x)$.
Prove that the number of lattice points in $S$ is equal to the sum
  $$\sum_{n=a}^b \floor{f(n)}.$$

\begin{proof}

  TODO

\end{proof}

\section{Exercise 7}%
\label{sec:exercise-7}

If $a$ and $b$ are positive integers with no common factor, we have the formula
  $$\sum_{n=1}^{b-1} \floor{\frac{na}{b}} = \frac{(a - 1)(b - 1)}{2}.$$
When $b = 1$, the sum on the left is understood to be $0$.

\subsection{Exercise 7a}%
\label{sub:exercise-7a}

Derive this result by a geometric argument, counting lattice points in a right
  triangle.

\begin{proof}

  TODO

\end{proof}

\subsection{Exercise 7b}%
\label{sub:exercise-7b}

Derive the result analytically as follows:
By changing the index of summation, note that
  $\sum_{n=1}^{b-1} \floor{na / b} = \sum_{n=1}^{b-1} \floor{a(b - n) / b}$.
Now apply Exercises 4(a) and (b) to the bracket on the right.

\begin{proof}

  \link{exercise\_7b}

\end{proof}

\section{Exercise 8}%
\label{sec:exercise-8}

Let $S$ be a set of points on the real line.
The \textit{characteristic function} of $S$ is, by definition, the function
  $\chi_S$ such that $\chi_S(x) = 1$ for every $x$ in $S$, and $\chi_S(x) = 0$
  for those $x$ not in $S$.
Let $f$ be a step function which takes the constant value $c_k$ on the $k$th
  open subinterval $I_k$ of some partition of an interval $[a, b]$.
Prove that for each $x$ in the union $I_1 \cup I_2 \cup \cdots \cup I_n$ we have
  $$f(x) = \sum_{k=1}^n c_k\chi_{I_k}(x).$$
This property is described by saying that every step function is a linear
  combination of characteristic functions of intervals.

\begin{proof}

  TODO

\end{proof}

\end{document}
