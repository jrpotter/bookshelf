\documentclass{article}

\usepackage{amsfonts, amsthm}
\usepackage{hyperref}

\newtheorem{theorem}{Theorem}
\newtheorem{xtheoreminner}{Theorem}
\newenvironment{xtheorem}[1]{%
  \renewcommand\thextheoreminner{#1}%
  \xtheoreminner
}{\endxtheoreminner}

\hypersetup{colorlinks=true, urlcolor=blue}

\newcommand{\ceil}[1]{\left\lceil#1\right\rceil}
\newcommand{\floor}[1]{\left\lfloor#1\right\rfloor}
% The first argument refers to a relative path upward from a current file to
% the root of the workspace (i.e. where this `preamble.tex` file is located).
\newcommand{\lean}[4]{\href{#1/#2.html\##3}{#4}}


\newcommand{\lean}[2]{\leanref{../../#1.html\##2}{#2}}

\begin{document}

\tableofcontents

\section{The Concepts of Integral Calculus}%
\label{sec:concepts-integral-calculus}

\subsection{\defined{Partition}}%
\label{sub:partition}

Let $[a, b]$ be a closed interval decomposed into $n$ subintervals by inserting
  $n - 1$ points of subdivision, say $x_1$, $x_2$, $\ldots$, $x_{n-1}$, subject
  only to the restriction
  \begin{equation}
    \label{sec:partition-eq1}
    a < x_1 < x_2 < \cdots < x_{n-1} < b.
  \end{equation}
It is convenient to denote the point $a$ itself by $x_0$ and the point $b$ by
  $x_n$.
A collection of points satisfying \eqref{sec:partition-eq1} is called a
  \textbf{partition} $P$ of $[a, b]$, and we use the symbol
  $$P = \{x_0, x_1, \ldots, x_n\}$$ to designate this partition.

\begin{definition}

  \lean{Common/Set/Intervals/Partition}{Set.Intervals.Partition}

\end{definition}

\subsection{\defined{Step Function}}%
\label{sub:step-function}

A function $s$, whose domain is a closed interval $[a, b]$, is called a step
  function if there is a \nameref{sub:partition} $P = \{x_0, x_1, \ldots, x_n\}$
  of $[a b]$ such that $s$ is constant on each open subinterval of $P$.
That is to say, for each $k = 1, 2, \ldots, n$, there is a real number $s_k$
  such that $$s(x) = s_k \quad\text{if}\quad x_{k-1} < x < x_k.$$
Step functions are sometimes called piecewise constant functions.

\vspace{8pt}
\noindent
\textit{Note:} At each of the endpoints $x_{k-1}$ and $x_k$ the function must
  have some well-defined value, but this need not be the same as $s_k$.

\begin{definition}

  \lean{Common/Set/Intervals/StepFunction}{Set.Intervals.StepFunction}

\end{definition}

\end{document}
