\documentclass{report}

\usepackage{amsfonts, amsthm}
\usepackage{hyperref}

\newtheorem{theorem}{Theorem}
\newtheorem{xtheoreminner}{Theorem}
\newenvironment{xtheorem}[1]{%
  \renewcommand\thextheoreminner{#1}%
  \xtheoreminner
}{\endxtheoreminner}

\hypersetup{colorlinks=true, urlcolor=blue}

\newcommand{\ceil}[1]{\left\lceil#1\right\rceil}
\newcommand{\floor}[1]{\left\lfloor#1\right\rfloor}
% The first argument refers to a relative path upward from a current file to
% the root of the workspace (i.e. where this `preamble.tex` file is located).
\newcommand{\lean}[4]{\href{#1/#2.html\##3}{#4}}

\makeleancommands{../..}

\begin{document}

\header{A Mathematical Introduction to Logic}{Herbert B. Enderton}

\tableofcontents

\begingroup
\renewcommand\thechapter{R}

\chapter{Reference}%
\hyperlabel{chap:reference}

\section{\defined{Construction Sequence}}%
\hyperlabel{ref:construction-sequence}

A \textbf{construction sequence} is a finite sequence
  $\langle \epsilon_1, \ldots, \epsilon_n \rangle$ of expressions such that for
  each $i \leq n$ we have at least one of
  \begin{align*}
    & \epsilon_i \text{ is a sentence symbol} \\
    & \epsilon_i = \mathcal{E}_\neg(\epsilon_j) \text{ for some } j < i \\
    & \epsilon_i = \mathcal{E}_\square(\epsilon_j, \epsilon_k)
      \text{ for some } j < i, k < i
  \end{align*}
  where $\square$ is one of the binary connectives $\land$, $\lor$,
    $\Rightarrow$, $\Leftrightarrow$.

\section{\defined{Expression}}%
\hyperlabel{ref:expression}

An \textbf{expression} is a finite sequence of symbols.

\section{\defined{Well-Formed Formula}}%
\hyperlabel{ref:well-formed-formula}

An \nameref{ref:expression} that can be built up from the sentence symbols by
  applying some finite number of times the \textbf{formula-building operations}
  (on expressions) defined by the equations:
  \begin{align*}
    \mathcal{E}_{\neg}(\alpha)
      & = (\neg \alpha) \\
    \mathcal{E}_{\land}(\alpha, \beta)
      & = (\alpha \land \beta) \\
    \mathcal{E}_{\lor}(\alpha, \beta)
      & = (\alpha \lor \beta) \\
    \mathcal{E}_{\Rightarrow}(\alpha, \beta)
      & = (\alpha \Rightarrow \beta) \\
    \mathcal{E}_{\Leftrightarrow}(\alpha, \beta)
      & = (\alpha \Leftrightarrow \beta)
  \end{align*}

\endgroup

% Reset counter to mirror Enderton's book.
\setcounter{chapter}{0}
\addtocounter{chapter}{-1}
\chapter{Useful Facts About Sets}%
\hyperlabel{chap:useful-facts-about-sets}

\section{\sorry{Lemma 0A}}%
\hyperlabel{sec:lemma-0a}

\begin{lemma}[0A]

Assume that $\langle x_1, \ldots, x_m \rangle =
  \langle y_1, \ldots, y_m, \ldots, y_{m+k} \rangle$.
Then $x_1 = \langle y_1, \ldots, y_{k+1} \rangle$.

\end{lemma}

\begin{proof}

  TODO

\end{proof}

\chapter{Sentential Logic}%
\hyperlabel{chap:sentential-logic}

\section{The Language of Sentential Logic}%
\hyperlabel{sec:language-sentential-logic}

\subsection{\sorry{Induction Principle}}%
\hyperlabel{sub:induction-principle-1}

\begin{theorem}

If $S$ is a set of wffs containing all the sentence symbols and closed under all
  five formula-building operations, then $S$ is the set of \textit{all} wffs.

\end{theorem}

\begin{proof}

  TODO

\end{proof}

\section{Exercises 1}%
\hyperlabel{sec:exercises-1}

\subsection{\sorry{Exercise 1.1}}%
\hyperlabel{sub:exercise-1.1}

Give three sentences in English together with translations into our formal
  language.
The sentences shoudl be chosen so as to have an interesting structure, and the
  translations should each contain 15 or more symbols.

\begin{answer}

  TODO

\end{answer}

\subsection{\sorry{Exercise 1.2}}%
\hyperlabel{sub:exercise-1.2}

Show that there are no wffs of length 2, 3, or 6, but that any other positive
  length is possible.

\begin{answer}

  TODO

\end{answer}

\subsection{\sorry{Exercise 1.3}}%
\hyperlabel{sub:exercise-1.3}

Let $\alpha$ be a wff; let $c$ be the number of places at which binary
  connective symbols $(\land, \lor, \Rightarrow, \Leftrightarrow)$ occur in
  $\alpha$; let $s$ be the number of places at which sentence symbols occur in
  $\alpha$.
(For exmaple, if $\alpha$ is $(A \Rightarrow (\neg A))$ then $c = 1$ and $s = 2$.)
Show by using the induction principle that $s = c + 1$.

\begin{answer}

  TODO

\end{answer}

\subsection{\sorry{Exercise 1.4}}%
\hyperlabel{sub:exercise-1.4}

Assume we have a construction sequence ending in $\phi$, where $\phi$ does not
  contain the symbol $A_4$.
Suppose we delete all the expressions in the construction sequence that contain
  $A_4$.
Show that the result is still a legal construction sequence.

\begin{answer}

  TODO

\end{answer}

\subsection{\sorry{Exercise 1.5}}%
\hyperlabel{sub:exercise-1.5}

Suppose that $\alpha$ is a wff not containing the negation symbol $\neg$.
\begin{enumerate}[(a)]
  \item Show that the length of $\alpha$ (i.e., the number of symbols in the
    string) is odd.
  \item Show that more than a quarter of the symbols are sentence symbols.
\end{enumerate}
\textit{Suggestion}: Apply induction to show that the length is of the form
  $4k + 1$ and the number of sentence symbols is $k + 1$.

\begin{answer}

  TODO

\end{answer}

\subsection{\sorry{Exercise 1.6}}%
\hyperlabel{sub:exercise-1.6}

Suppose that $\alpha$ is a wff not containing the negation symbol $\neg$.
\begin{enumerate}[(a)]
  \item Show that the length of $\alpha$ (i.e., the number of symbols in the
    string) is odd.
  \item Show that more than a quarter of the symbols are sentence symbols.
\end{enumerate}

\begin{answer}

  TODO

\end{answer}

\end{document}
