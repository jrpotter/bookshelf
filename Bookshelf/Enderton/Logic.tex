\documentclass{report}

\usepackage{amsfonts, amsmath, amssymb, amsthm}
\usepackage{bigfoot}
\usepackage{comment}
\usepackage[shortlabels]{enumitem}
\usepackage{etoolbox}
\usepackage{environ}
\usepackage{fontawesome5}
\usepackage{mathabx, mathrsfs}
\usepackage{soul}
\usepackage{stmaryrd}
% Must load `xcolor` before `tcolorbox` and `tikz`.
\usepackage[dvipsnames]{xcolor}
\usepackage{tcolorbox}
\usepackage{tikz}
% `hyperref` comes after `xr-hyper`.
\usepackage{xr-hyper}
\usepackage{hyperref}

% Open "private" namespace.
\makeatletter

% ========================================
% General
% ========================================

\newcommand{\header}[2]{\title{#1}\author{#2}\date{}\maketitle}

% ========================================
% Dividers
% ========================================

\newcommand\@linespace{\vspace{10pt}}
\newcommand\linedivider{\@linespace\hrule\@linespace}
\WithSuffix\newcommand\linedivider*{\@linespace\hrule}
\newcommand\suitdivider{$$\spadesuit\;\spadesuit\;\spadesuit$$}

% ========================================
% Linking
% ========================================

\hypersetup{colorlinks=true, linkcolor=blue, urlcolor=blue}
\newcommand{\textref}[1]{\text{\nameref{#1}}}
\newcommand{\hyperlabel}[1]{%
  \label{#1}%
  \hypertarget{#1}{}}

% Links to theorems/statements/etc. that can be found in Mathlib4's index.
\newcommand\@leanlink[3]{%
  \textcolor{BlueViolet}{\raisebox{-4.5pt}{%
    \tikz{\draw (0, 0) node[yscale=-1,xscale=1] {\faFont};}}{-\;}}%
  \href{https://leanprover-community.github.io/mathlib4_docs/#1.html\##2}%
  {\color{BlueViolet}{#3}}}

\newcommand\lean[2]{%
  \noindent\@leanlink{#1}{#2}{#2}}
\WithSuffix\newcommand\lean*[2]{%
  \vspace{6pt}\lean{#1}{#2}}

\newcommand\leanp[3]{%
  \noindent\@leanlink{#1}{#2}{#3}}
\WithSuffix\newcommand\leanp*[3]{%
  \vspace{6pt}\leanp{#1}{#2}{#3}}

% Links to theorems/statements/etc. found in custom index.
\newcommand\@codelink[4]{%
  \textcolor{MidnightBlue}{\raisebox{-4.5pt}{%
    \tikz{\draw (0, 0) node[xshift=8pt] {\faCodeBranch};}}{-\;}}%
  \href{#1/#2.html\##3}%
  {\color{MidnightBlue}{#4}}}

\newcommand\coderef[3]{%
  \@codelink{#1}{#2}{#3}{#3}}
\newcommand\codepref[4]{%
  \@codelink{#1}{#2}{#3}{#4}}

% Macro to build our `code` commands relative to a given directory. For
% instance, we expect to have invocation `\makecode{..}` if the TeX file exists
% one directory deep from the root of our project..
\newcommand\makecode[1]{%
  \newcommand\code[2]{%
    \noindent\coderef{#1}{##1}{##2}}
  \WithSuffix\newcommand\code*[2]{%
    \vspace{6pt}\noindent\coderef{#1}{##1}{##2}}

  \newcommand\codep[3]{%
    \noindent\codepref{#1}{##1}{##2}{##3}}
  \WithSuffix\newcommand\codep*[3]{%
    \vspace{6pt}\noindent\codepref{#1}{##1}{##2}{##3}}
}

% ========================================
% Admonitions
% ========================================

\NewEnviron{note}{%
  \begin{tcolorbox}[%
      sharp corners,
      fonttitle=\sffamily\bfseries,
      toptitle=2pt,
      bottomtitle=2pt,
      coltitle=black!80!white,
      colback=yellow!30,
      colframe=yellow!80!black,
      title=Note]
    \BODY
  \end{tcolorbox}}

% ========================================
% Statements
% ========================================

\newcommand\@statement[1]{%
  \linedivider*\paragraph{\normalfont\normalsize\textit{#1.}}}
\newenvironment{answer}{\@statement{Answer}}{\hfill$\square$}
\renewenvironment{proof}{\@statement{Proof}}{\hfill$\square$}

\newtheorem{corollaryinner}{Corollary}
\newenvironment{corollary}[1][]{%
  \ifstrempty{#1}
    {\corollaryinner}
    {\renewcommand\thecorollaryinner{#1}\corollaryinner}
}{\endcorollaryinner}

\newtheorem{lemmainner}{Lemma}
\newenvironment{lemma}[1][]{%
  \ifstrempty{#1}
    {\lemmainner}
    {\renewcommand\thelemmainner{#1}\lemmainner}
}{\endlemmainner}

\newtheorem{theoreminner}{Theorem}
\newenvironment{theorem}[1][]{%
  \ifstrempty{#1}
    {\theoreminner}
    {\renewcommand\thetheoreminner{#1}\theoreminner}
}{\endtheoreminner}

% ========================================
% Status
% ========================================

\DeclareRobustCommand{\defined}[1]{%
  \texorpdfstring{\color{darkgray}\faParagraph\ #1}{#1}}
\DeclareRobustCommand{\verified}[1]{%
  \texorpdfstring{\color{teal}\faCheckCircle\ #1}{#1}}
\DeclareRobustCommand{\unverified}[1]{%
  \texorpdfstring{\color{olive}\faCheckCircle[regular]\ #1}{#1}}
\DeclareRobustCommand{\pending}[1]{%
  \texorpdfstring{\color{Fuchsia}\faPencil*\ #1}{#1}}
\DeclareRobustCommand{\sorry}[1]{%
  \texorpdfstring{\color{Maroon}\faExclamationCircle\ #1}{#1}}

% ========================================
% Math
% ========================================

\newcommand{\abs}[1]{\left|#1\right|}
\newcommand{\ceil}[1]{\left\lceil#1\right\rceil}
\newcommand{\dom}[1]{\textop{dom}{#1}}
\newcommand{\fld}[1]{\textop{fld}{#1}}
\newcommand{\floor}[1]{\left\lfloor#1\right\rfloor}
\newcommand{\icc}[2]{\left[#1, #2\right]}
\newcommand{\ico}[2]{\left[#1, #2\right)}
\newcommand{\img}[2]{#1\!\left\llbracket#2\right\rrbracket}
\newcommand{\ioc}[2]{\left(#1, #2\right]}
\newcommand{\ioo}[2]{\left(#1, #2\right)}
\newcommand{\powerset}[1]{\mathscr{P}#1}
\newcommand{\ran}[1]{\textop{ran}{#1}}
\newcommand{\textop}[1]{\mathop{\text{#1}}}
\newcommand{\ubar}[1]{\text{\b{$#1$}}}

\let\oldemptyset\emptyset
\let\emptyset\varnothing

% Close off "private" namespace.
\makeatother

\makeleancommands{../..}

\begin{document}

\header{A Mathematical Introduction to Logic}{Herbert B. Enderton}

\tableofcontents

\begingroup
\renewcommand\thechapter{R}

\chapter{Reference}%
\hyperlabel{chap:reference}

\section{\defined{Construction Sequence}}%
\hyperlabel{ref:construction-sequence}

  A \textbf{construction sequence} is a finite sequence
    $\langle \epsilon_1, \ldots, \epsilon_n \rangle$ of expressions such that
    for each $i \leq n$ we have at least one of
    \begin{align*}
      & \epsilon_i \text{ is a sentence symbol} \\
      & \epsilon_i = \mathcal{E}_\neg(\epsilon_j) \text{ for some } j < i \\
      & \epsilon_i = \mathcal{E}_\square(\epsilon_j, \epsilon_k)
        \text{ for some } j < i, k < i
    \end{align*}
    where $\square$ is one of the binary connectives $\land$, $\lor$,
      $\Rightarrow$, $\Leftrightarrow$.

\section{\defined{Expression}}%
\hyperlabel{ref:expression}

  An \textbf{expression} is a finite sequence of symbols.

\section{\defined{Well-Formed Formula}}%
\hyperlabel{ref:well-formed-formula}

  An \nameref{ref:expression} that can be built up from the sentence symbols by
    applying some finite number of times the
    \textbf{formula-building operations} (on expressions) defined by the
    equations:
    \begin{align*}
      \mathcal{E}_{\neg}(\alpha)
        & = (\neg \alpha) \\
      \mathcal{E}_{\land}(\alpha, \beta)
        & = (\alpha \land \beta) \\
      \mathcal{E}_{\lor}(\alpha, \beta)
        & = (\alpha \lor \beta) \\
      \mathcal{E}_{\Rightarrow}(\alpha, \beta)
        & = (\alpha \Rightarrow \beta) \\
      \mathcal{E}_{\Leftrightarrow}(\alpha, \beta)
        & = (\alpha \Leftrightarrow \beta)
    \end{align*}

\endgroup

% Reset counter to mirror Enderton's book.
\setcounter{chapter}{0}
\addtocounter{chapter}{-1}
\chapter{Useful Facts About Sets}%
\hyperlabel{chap:useful-facts-about-sets}

\section{\sorry{Lemma 0A}}%
\hyperlabel{sec:lemma-0a}

  \begin{lemma}[0A]
    Assume that $\langle x_1, \ldots, x_m \rangle =
      \langle y_1, \ldots, y_m, \ldots, y_{m+k} \rangle$.
    Then $x_1 = \langle y_1, \ldots, y_{k+1} \rangle$.
  \end{lemma}

  \begin{proof}
    TODO
  \end{proof}

\chapter{Sentential Logic}%
\hyperlabel{chap:sentential-logic}

\section{The Language of Sentential Logic}%
\hyperlabel{sec:language-sentential-logic}

\subsection{\sorry{Induction Principle}}%
\hyperlabel{sub:induction-principle-1}

  \begin{theorem}
    If $S$ is a set of wffs containing all the sentence symbols and closed under
      all five formula-building operations, then $S$ is the set of \textit{all}
      wffs.
  \end{theorem}

  \begin{proof}
    TODO
  \end{proof}

\section{Exercises 1}%
\hyperlabel{sec:exercises-1}

\subsection{\sorry{Exercise 1.1.1}}%
\hyperlabel{sub:exercise-1.1.1}

  Give three sentences in English together with translations into our formal
    language.
  The sentences shoudl be chosen so as to have an interesting structure, and the
    translations should each contain 15 or more symbols.

  \begin{answer}
    TODO
  \end{answer}

\subsection{\sorry{Exercise 1.1.2}}%
\hyperlabel{sub:exercise-1.1.2}

  Show that there are no wffs of length 2, 3, or 6, but that any other positive
    length is possible.

  \begin{proof}
    TODO
  \end{proof}

\subsection{\sorry{Exercise 1.1.3}}%
\hyperlabel{sub:exercise-1.1.3}

  Let $\alpha$ be a wff; let $c$ be the number of places at which binary
    connective symbols $(\land, \lor, \Rightarrow, \Leftrightarrow)$ occur in
    $\alpha$; let $s$ be the number of places at which sentence symbols occur in
    $\alpha$.
  (For example, if $\alpha$ is $(A \Rightarrow (\neg A))$ then $c = 1$ and
    $s = 2$.)
  Show by using the induction principle that $s = c + 1$.

  \begin{proof}
    TODO
  \end{proof}

\subsection{\sorry{Exercise 1.1.4}}%
\hyperlabel{sub:exercise-1.1.4}

  Assume we have a construction sequence ending in $\phi$, where $\phi$ does not
    contain the symbol $A_4$.
  Suppose we delete all the expressions in the construction sequence that
    contain $A_4$.
  Show that the result is still a legal construction sequence.

  \begin{proof}
    TODO
  \end{proof}

\subsection{\sorry{Exercise 1.1.5}}%
\hyperlabel{sub:exercise-1.1.5}

  Suppose that $\alpha$ is a wff not containing the negation symbol $\neg$.
  \begin{enumerate}[(a)]
    \item Show that the length of $\alpha$ (i.e., the number of symbols in the
      string) is odd.
    \item Show that more than a quarter of the symbols are sentence symbols.
  \end{enumerate}
  \textit{Suggestion}: Apply induction to show that the length is of the form
    $4k + 1$ and the number of sentence symbols is $k + 1$.

  \begin{proof}
    TODO
  \end{proof}

\subsection{\sorry{Exercise 1.1.6}}%
\hyperlabel{sub:exercise-1.1.6}

  Suppose that $\alpha$ is a wff not containing the negation symbol $\neg$.
  \begin{enumerate}[(a)]
    \item Show that the length of $\alpha$ (i.e., the number of symbols in the
      string) is odd.
    \item Show that more than a quarter of the symbols are sentence symbols.
  \end{enumerate}

  \begin{proof}
    TODO
  \end{proof}

\end{document}
