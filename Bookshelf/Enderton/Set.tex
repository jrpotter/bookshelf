\documentclass{report}

\usepackage{amsfonts, amsthm}
\usepackage{hyperref}

\newtheorem{theorem}{Theorem}
\newtheorem{xtheoreminner}{Theorem}
\newenvironment{xtheorem}[1]{%
  \renewcommand\thextheoreminner{#1}%
  \xtheoreminner
}{\endxtheoreminner}

\hypersetup{colorlinks=true, urlcolor=blue}

\newcommand{\ceil}[1]{\left\lceil#1\right\rceil}
\newcommand{\floor}[1]{\left\lfloor#1\right\rfloor}
% The first argument refers to a relative path upward from a current file to
% the root of the workspace (i.e. where this `preamble.tex` file is located).
\newcommand{\lean}[4]{\href{#1/#2.html\##3}{#4}}

\makeleancommands{../..}

\begin{document}

\header{Elements of Set Theory}{Herbert B. Enderton}

\tableofcontents

\begingroup
\renewcommand\thechapter{R}
\setcounter{chapter}{0}
\addtocounter{chapter}{-1}

\chapter{Reference}%
\label{chap:reference}

\section{\defined{Powerset}}%
\label{ref:powerset}

The \textbf{powerset} of some set $A$ is the set of all subsets of $A$.

\begin{definition}

  \lean{Mathlib/Init/Set}{Set.powerset}

\end{definition}

\section{\defined{Principle of Extensionality}}%
\label{ref:principle-extensionality}

If $A$ and $B$ are sets such that for every object $t$,
  $$t \in A \quad\text{iff}\quad t \in B,$$
  then $A = B$.

\begin{axiom}

  \lean{Mathlib/Init/Set}{Set.ext}

\end{axiom}

\endgroup

\chapter{Introduction}%
\label{chap:introduction}

\section{Baby Set Theory}%
\label{sec:baby-set-theory}

\subsection{\partial{Exercise 1}}%
\label{sub:baby-set-theory-1}

Which of the following become true when "$\in$" is inserted in place of the
  blank?
Which become true when "$\subseteq$" is inserted?

\subsubsection{\partial{Exercise 1a}}%
\label{ssub:baby-set-theory-1a}

$\{\emptyset\} \_\_\_\_ \{\emptyset, \{\emptyset\}\}$.

\begin{proof}

  Because the \textit{object} $\{\emptyset\}$ is a member of the right-hand set,
    the statement is \textbf{true} in the case of "$\in$".

  Because the \textit{members} of $\{\emptyset\}$ are all members of the
    right-hand set, the statement is also \textbf{true} in the case of
    "$\subseteq$".

\end{proof}

\subsubsection{\partial{Exercise 1b}}%
\label{ssub:baby-set-theory-1b}

$\{\emptyset\} \_\_\_\_ \{\emptyset, \{\{\emptyset\}\}\}$.

\begin{proof}

  Because the \textit{object} $\{\emptyset\}$ is not a member of the right-hand
    set, the statement is \textbf{false} in the case of "$\in$".

  Because the \textit{members} of $\{\emptyset\}$ are all members of the
    right-hand set, the statement is \textbf{true} in the case of "$\subseteq$".

\end{proof}

\subsubsection{\partial{Exercise 1c}}%
\label{ssub:baby-set-theory-1c}

$\{\{\emptyset\}\} \_\_\_\_ \{\emptyset, \{\emptyset\}\}$.

\begin{proof}

  Because the \textit{object} $\{\{\emptyset\}\}$ is not a member of the
    right-hand set, the statement is \textbf{false} in the case of "$\in$".

  Because the \textit{members} of $\{\{\emptyset\}\}$ are all members of the
    right-hand set, the statement is \textbf{true} in the case of "$\subseteq$".

\end{proof}

\subsubsection{\partial{Exercise 1d}}%
\label{ssub:baby-set-theory-1d}

$\{\{\emptyset\}\} \_\_\_\_ \{\emptyset, \{\{\emptyset\}\}\}$.

\begin{proof}

  Because the \textit{object} $\{\{\emptyset\}\}$ is a member of the right-hand
    set, the statement is \textbf{true} in the case of "$\in$".

  Because the \textit{members} of $\{\{\emptyset\}\}$ are not all members of the
    right-hand set, the statement is \textbf{false} in the case of
    "$\subseteq$".

\end{proof}

\subsubsection{\partial{Exercise 1e}}%
\label{ssub:baby-set-theory-1e}

$\{\{\emptyset\}\} \_\_ \{\emptyset, \{\emptyset, \{\emptyset\}\}\}$.

\begin{proof}

  Because the \textit{object} $\{\{\emptyset\}\}$ is not a member of the
    right-hand set, the statement is \textbf{false} in the case of "$\in$".

  Because the \textit{members} of $\{\{\emptyset\}\}$ are not all members of the
    right-hand set, the statement is \textbf{false} in the case of
    "$\subseteq$".

\end{proof}

\subsection{\partial{Exercise 2}}%
\label{sub:baby-set-theory-2}

Show that no two of the three sets $\emptyset$, $\{\emptyset\}$, and
  $\{\{\emptyset\}\}$ are equal to each other.

\begin{proof}

  By the \nameref{ref:principle-extensionality}, $\emptyset$ is only equal to
    $\emptyset$.
  This immediately shows it is not equal to the other two.
  Now consider object $\emptyset$.
  This object is a member of $\{\emptyset\}$ but is not a member of
    $\{\{\emptyset\}\}$.
  Again, by the \nameref{ref:principle-extensionality}, these two sets must be
    different.

\end{proof}

\subsection{\partial{Exercise 3}}%
\label{sub:baby-set-theory-3}

Show that if $B \subseteq C$, then $\mathscr{P} B \subseteq \mathscr{P} C$.

\begin{proof}

  Let $x \in \mathscr{P} B$.
  By definition of the \nameref{ref:powerset}, $x$ is a subset of $B$.
  By hypothesis, $B \subseteq C$.
  Then $x \subseteq C$.
  Again by definition of the \nameref{ref:powerset}, it follows
    $x \in \mathscr{P} C$.

\end{proof}

\subsection{\partial{Exercise 4}}%
\label{sub:baby-set-theory-4}

Assume that $x$ and $y$ are members of a set $B$.
Show that $\{\{x\}, \{x, y\}\} \in \mathscr{P}\mathscr{P} B.$

\begin{proof}

  Let $x$ and $y$ be members of set $B$.
  Then $\{x\}$ and $\{x, y\}$ are subsets of $B$.
  By definition of the \nameref{ref:powerset}, $\{x\}$ and $\{x, y\}$ are
    members of $\mathscr{P} B$.
  Then $\{\{x\}, \{x, y\}\}$ is a subset of $\mathscr{P} B$.
  By definition of the \nameref{ref:powerset}, $\{\{x\}, \{x, y\}\}$ is a member
    of $\mathscr{P}\mathscr{P} B$.

\end{proof}

\end{document}
