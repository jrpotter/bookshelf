\documentclass{report}

\usepackage{amsfonts, amsmath, amssymb, amsthm}
\usepackage{bigfoot}
\usepackage{comment}
\usepackage[shortlabels]{enumitem}
\usepackage{etoolbox}
\usepackage{environ}
\usepackage{fontawesome5}
\usepackage{mathabx, mathrsfs}
\usepackage{soul}
\usepackage{stmaryrd}
% Must load `xcolor` before `tcolorbox` and `tikz`.
\usepackage[dvipsnames]{xcolor}
\usepackage{tcolorbox}
\usepackage{tikz}
% `hyperref` comes after `xr-hyper`.
\usepackage{xr-hyper}
\usepackage{hyperref}

% Open "private" namespace.
\makeatletter

% ========================================
% General
% ========================================

\newcommand{\header}[2]{\title{#1}\author{#2}\date{}\maketitle}

% ========================================
% Dividers
% ========================================

\newcommand\@linespace{\vspace{10pt}}
\newcommand\linedivider{\@linespace\hrule\@linespace}
\WithSuffix\newcommand\linedivider*{\@linespace\hrule}
\newcommand\suitdivider{$$\spadesuit\;\spadesuit\;\spadesuit$$}

% ========================================
% Linking
% ========================================

\hypersetup{colorlinks=true, linkcolor=blue, urlcolor=blue}
\newcommand{\textref}[1]{\text{\nameref{#1}}}
\newcommand{\hyperlabel}[1]{%
  \label{#1}%
  \hypertarget{#1}{}}

% Links to theorems/statements/etc. that can be found in Mathlib4's index.
\newcommand\@leanlink[3]{%
  \textcolor{BlueViolet}{\raisebox{-4.5pt}{%
    \tikz{\draw (0, 0) node[yscale=-1,xscale=1] {\faFont};}}{-\;}}%
  \href{https://leanprover-community.github.io/mathlib4_docs/#1.html\##2}%
  {\color{BlueViolet}{#3}}}

\newcommand\lean[2]{%
  \noindent\@leanlink{#1}{#2}{#2}}
\WithSuffix\newcommand\lean*[2]{%
  \vspace{6pt}\lean{#1}{#2}}

\newcommand\leanp[3]{%
  \noindent\@leanlink{#1}{#2}{#3}}
\WithSuffix\newcommand\leanp*[3]{%
  \vspace{6pt}\leanp{#1}{#2}{#3}}

% Links to theorems/statements/etc. found in custom index.
\newcommand\@codelink[4]{%
  \textcolor{MidnightBlue}{\raisebox{-4.5pt}{%
    \tikz{\draw (0, 0) node[xshift=8pt] {\faCodeBranch};}}{-\;}}%
  \href{#1/#2.html\##3}%
  {\color{MidnightBlue}{#4}}}

\newcommand\coderef[3]{%
  \@codelink{#1}{#2}{#3}{#3}}
\newcommand\codepref[4]{%
  \@codelink{#1}{#2}{#3}{#4}}

% Macro to build our `code` commands relative to a given directory. For
% instance, we expect to have invocation `\makecode{..}` if the TeX file exists
% one directory deep from the root of our project..
\newcommand\makecode[1]{%
  \newcommand\code[2]{%
    \noindent\coderef{#1}{##1}{##2}}
  \WithSuffix\newcommand\code*[2]{%
    \vspace{6pt}\noindent\coderef{#1}{##1}{##2}}

  \newcommand\codep[3]{%
    \noindent\codepref{#1}{##1}{##2}{##3}}
  \WithSuffix\newcommand\codep*[3]{%
    \vspace{6pt}\noindent\codepref{#1}{##1}{##2}{##3}}
}

% ========================================
% Admonitions
% ========================================

\NewEnviron{note}{%
  \begin{tcolorbox}[%
      sharp corners,
      fonttitle=\sffamily\bfseries,
      toptitle=2pt,
      bottomtitle=2pt,
      coltitle=black!80!white,
      colback=yellow!30,
      colframe=yellow!80!black,
      title=Note]
    \BODY
  \end{tcolorbox}}

% ========================================
% Statements
% ========================================

\newcommand\@statement[1]{%
  \linedivider*\paragraph{\normalfont\normalsize\textit{#1.}}}
\newenvironment{answer}{\@statement{Answer}}{\hfill$\square$}
\renewenvironment{proof}{\@statement{Proof}}{\hfill$\square$}

\newtheorem{corollaryinner}{Corollary}
\newenvironment{corollary}[1][]{%
  \ifstrempty{#1}
    {\corollaryinner}
    {\renewcommand\thecorollaryinner{#1}\corollaryinner}
}{\endcorollaryinner}

\newtheorem{lemmainner}{Lemma}
\newenvironment{lemma}[1][]{%
  \ifstrempty{#1}
    {\lemmainner}
    {\renewcommand\thelemmainner{#1}\lemmainner}
}{\endlemmainner}

\newtheorem{theoreminner}{Theorem}
\newenvironment{theorem}[1][]{%
  \ifstrempty{#1}
    {\theoreminner}
    {\renewcommand\thetheoreminner{#1}\theoreminner}
}{\endtheoreminner}

% ========================================
% Status
% ========================================

\DeclareRobustCommand{\defined}[1]{%
  \texorpdfstring{\color{darkgray}\faParagraph\ #1}{#1}}
\DeclareRobustCommand{\verified}[1]{%
  \texorpdfstring{\color{teal}\faCheckCircle\ #1}{#1}}
\DeclareRobustCommand{\unverified}[1]{%
  \texorpdfstring{\color{olive}\faCheckCircle[regular]\ #1}{#1}}
\DeclareRobustCommand{\pending}[1]{%
  \texorpdfstring{\color{Fuchsia}\faPencil*\ #1}{#1}}
\DeclareRobustCommand{\sorry}[1]{%
  \texorpdfstring{\color{Maroon}\faExclamationCircle\ #1}{#1}}

% ========================================
% Math
% ========================================

\newcommand{\abs}[1]{\left|#1\right|}
\newcommand{\ceil}[1]{\left\lceil#1\right\rceil}
\newcommand{\dom}[1]{\textop{dom}{#1}}
\newcommand{\fld}[1]{\textop{fld}{#1}}
\newcommand{\floor}[1]{\left\lfloor#1\right\rfloor}
\newcommand{\icc}[2]{\left[#1, #2\right]}
\newcommand{\ico}[2]{\left[#1, #2\right)}
\newcommand{\img}[2]{#1\!\left\llbracket#2\right\rrbracket}
\newcommand{\ioc}[2]{\left(#1, #2\right]}
\newcommand{\ioo}[2]{\left(#1, #2\right)}
\newcommand{\powerset}[1]{\mathscr{P}#1}
\newcommand{\ran}[1]{\textop{ran}{#1}}
\newcommand{\textop}[1]{\mathop{\text{#1}}}
\newcommand{\ubar}[1]{\text{\b{$#1$}}}

\let\oldemptyset\emptyset
\let\emptyset\varnothing

% Close off "private" namespace.
\makeatother

\makeleancommands{../..}

\begin{document}

\header{Elements of Set Theory}{Herbert B. Enderton}

\tableofcontents

\begingroup
\renewcommand\thechapter{R}
\setcounter{chapter}{0}
\addtocounter{chapter}{-1}

\chapter{Reference}%
\label{chap:reference}

\section{\defined{Powerset}}%
\label{ref:powerset}

The \textbf{powerset} of some set $A$ is the set of all subsets of $A$.

\begin{definition}

  \lean{Mathlib/Init/Set}{Set.powerset}

\end{definition}

\section{\defined{Principle of Extensionality}}%
\label{ref:principle-extensionality}

If $A$ and $B$ are sets such that for every object $t$,
  $$t \in A \quad\text{iff}\quad t \in B,$$
  then $A = B$.

\begin{axiom}

  \lean{Mathlib/Init/Set}{Set.ext}

\end{axiom}

\endgroup

\chapter{Introduction}%
\label{chap:introduction}

\section{Baby Set Theory}%
\label{sec:baby-set-theory}

\subsection{\verified{Exercise 1}}%
\label{sub:baby-set-theory-1}

Which of the following become true when "$\in$" is inserted in place of the
  blank?
Which become true when "$\subseteq$" is inserted?

\subsubsection{\verified{Exercise 1a}}%
\label{ssub:baby-set-theory-1a}

$\{\emptyset\} \_\_\_\_ \{\emptyset, \{\emptyset\}\}$.

\begin{proof}

  \lean{Bookshelf/Enderton/Set/Chapter\_1}
    {Enderton.Set.Chapter\_1.exercise\_1a}

  Because the \textit{object} $\{\emptyset\}$ is a member of the right-hand set,
    the statement is \textbf{true} in the case of "$\in$".

  Because the \textit{members} of $\{\emptyset\}$ are all members of the
    right-hand set, the statement is also \textbf{true} in the case of
    "$\subseteq$".

\end{proof}

\subsubsection{\verified{Exercise 1b}}%
\label{ssub:baby-set-theory-1b}

$\{\emptyset\} \_\_\_\_ \{\emptyset, \{\{\emptyset\}\}\}$.

\begin{proof}

  \lean{Bookshelf/Enderton/Set/Chapter\_1}
    {Enderton.Set.Chapter\_1.exercise\_1b}

  Because the \textit{object} $\{\emptyset\}$ is not a member of the right-hand
    set, the statement is \textbf{false} in the case of "$\in$".

  Because the \textit{members} of $\{\emptyset\}$ are all members of the
    right-hand set, the statement is \textbf{true} in the case of "$\subseteq$".

\end{proof}

\subsubsection{\verified{Exercise 1c}}%
\label{ssub:baby-set-theory-1c}

$\{\{\emptyset\}\} \_\_\_\_ \{\emptyset, \{\emptyset\}\}$.

\begin{proof}

  \lean{Bookshelf/Enderton/Set/Chapter\_1}
    {Enderton.Set.Chapter\_1.exercise\_1c}

  Because the \textit{object} $\{\{\emptyset\}\}$ is not a member of the
    right-hand set, the statement is \textbf{false} in the case of "$\in$".

  Because the \textit{members} of $\{\{\emptyset\}\}$ are all members of the
    right-hand set, the statement is \textbf{true} in the case of "$\subseteq$".

\end{proof}

\subsubsection{\verified{Exercise 1d}}%
\label{ssub:baby-set-theory-1d}

$\{\{\emptyset\}\} \_\_\_\_ \{\emptyset, \{\{\emptyset\}\}\}$.

\begin{proof}

  \lean{Bookshelf/Enderton/Set/Chapter\_1}
    {Enderton.Set.Chapter\_1.exercise\_1d}

  Because the \textit{object} $\{\{\emptyset\}\}$ is a member of the right-hand
    set, the statement is \textbf{true} in the case of "$\in$".

  Because the \textit{members} of $\{\{\emptyset\}\}$ are not all members of the
    right-hand set, the statement is \textbf{false} in the case of
    "$\subseteq$".

\end{proof}

\subsubsection{\verified{Exercise 1e}}%
\label{ssub:baby-set-theory-1e}

$\{\{\emptyset\}\} \_\_ \{\emptyset, \{\emptyset, \{\emptyset\}\}\}$.

\begin{proof}

  \lean{Bookshelf/Enderton/Set/Chapter\_1}
    {Enderton.Set.Chapter\_1.exercise\_1e}

  Because the \textit{object} $\{\{\emptyset\}\}$ is not a member of the
    right-hand set, the statement is \textbf{false} in the case of "$\in$".

  Because the \textit{members} of $\{\{\emptyset\}\}$ are not all members of the
    right-hand set, the statement is \textbf{false} in the case of
    "$\subseteq$".

\end{proof}

\subsection{\verified{Exercise 2}}%
\label{sub:baby-set-theory-2}

Show that no two of the three sets $\emptyset$, $\{\emptyset\}$, and
  $\{\{\emptyset\}\}$ are equal to each other.

\begin{proof}

  \lean{Bookshelf/Enderton/Set/Chapter\_1}
    {Enderton.Set.Chapter\_1.exercise\_2}

  By the \nameref{ref:principle-extensionality}, $\emptyset$ is only equal to
    $\emptyset$.
  This immediately shows it is not equal to the other two.
  Now consider object $\emptyset$.
  This object is a member of $\{\emptyset\}$ but is not a member of
    $\{\{\emptyset\}\}$.
  Again, by the \nameref{ref:principle-extensionality}, these two sets must be
    different.

\end{proof}

\subsection{\verified{Exercise 3}}%
\label{sub:baby-set-theory-3}

Show that if $B \subseteq C$, then $\mathscr{P} B \subseteq \mathscr{P} C$.

\begin{proof}

  \lean{Bookshelf/Enderton/Set/Chapter\_1}
    {Enderton.Set.Chapter\_1.exercise\_3}

  Let $x \in \mathscr{P} B$.
  By definition of the \nameref{ref:powerset}, $x$ is a subset of $B$.
  By hypothesis, $B \subseteq C$.
  Then $x \subseteq C$.
  Again by definition of the \nameref{ref:powerset}, it follows
    $x \in \mathscr{P} C$.

\end{proof}

\subsection{\verified{Exercise 4}}%
\label{sub:baby-set-theory-4}

Assume that $x$ and $y$ are members of a set $B$.
Show that $\{\{x\}, \{x, y\}\} \in \mathscr{P}\mathscr{P} B.$

\begin{proof}

  \lean{Bookshelf/Enderton/Set/Chapter\_1}
    {Enderton.Set.Chapter\_1.exercise\_4}

  Let $x$ and $y$ be members of set $B$.
  Then $\{x\}$ and $\{x, y\}$ are subsets of $B$.
  By definition of the \nameref{ref:powerset}, $\{x\}$ and $\{x, y\}$ are
    members of $\mathscr{P} B$.
  Then $\{\{x\}, \{x, y\}\}$ is a subset of $\mathscr{P} B$.
  By definition of the \nameref{ref:powerset}, $\{\{x\}, \{x, y\}\}$ is a member
    of $\mathscr{P}\mathscr{P} B$.

\end{proof}

\end{document}
