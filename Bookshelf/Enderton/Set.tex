\documentclass{report}

\usepackage{amsfonts, amsthm}
\usepackage{hyperref}

\newtheorem{theorem}{Theorem}
\newtheorem{xtheoreminner}{Theorem}
\newenvironment{xtheorem}[1]{%
  \renewcommand\thextheoreminner{#1}%
  \xtheoreminner
}{\endxtheoreminner}

\hypersetup{colorlinks=true, urlcolor=blue}

\newcommand{\ceil}[1]{\left\lceil#1\right\rceil}
\newcommand{\floor}[1]{\left\lfloor#1\right\rfloor}
% The first argument refers to a relative path upward from a current file to
% the root of the workspace (i.e. where this `preamble.tex` file is located).
\newcommand{\lean}[4]{\href{#1/#2.html\##3}{#4}}

\makeleancommands{../..}

\begin{document}

\header{Elements of Set Theory}{Herbert B. Enderton}

\tableofcontents

\begingroup
\renewcommand\thechapter{R}
\setcounter{chapter}{0}
\addtocounter{chapter}{-1}

\chapter{Reference}%
\label{chap:reference}

\section{\defined{Empty Set Axiom}}%
\label{ref:empty-set-axiom}

There is a set having no members:
  $$\exists B, \forall x, x \not\in B.$$

\begin{axiom}

  \lean*{Mathlib/Init/Set}{Set.emptyCollection}

\end{axiom}

\section{\defined{Extensionality Axiom}}%
\label{ref:extensionality-axiom}

If two sets have exactly the same members, then they are equal:
  $$\forall A, \forall B,
      \left[\forall x, (x \in A \iff x \in B) \Rightarrow A = B\right].$$

\begin{axiom}

  \lean*{Mathlib/Init/Set}{Set.ext}

\end{axiom}

\section{\defined{Pair Set}}%
\label{ref:pair-set}

For any sets $u$ and $v$, the \textbf{pair set $\{u, v\}$} is the set whose
  only members are $u$ and $v$.

\begin{definition}

  \ % Add space

  \lean*{Mathlib/Init/Set}{Set.insert}

  \lean*{Mathlib/Init/Set}{Set.singleton}

\end{definition}

\section{\defined{Pairing Axiom}}%
\label{ref:pairing-axiom}

For any sets $u$ and $v$, there is a set having as members just $u$ and $v$:
  $$\forall u, \forall v, \exists B, \forall x,
      (x \in B \iff x = u \text{ or } x = v).$$

\begin{axiom}

  \ % Add space

  \lean*{Mathlib/Init/Set}{Set.insert}

  \lean*{Mathlib/Init/Set}{Set.singleton}

\end{axiom}

\section{\defined{Power Set}}%
\label{ref:power-set}

For any set $a$, the \textbf{power set $\powerset{a}$} is the set whose members
  are exactly the subsets of $a$.

\begin{definition}

  \lean*{Mathlib/Init/Set}{Set.powerset}

\end{definition}

\section{\defined{Power Set Axiom}}%
\label{ref:power-set-axiom}

For any set $a$, there is a set whose members are exactly the subsets of $a$:
  $$\forall a, \exists B, \forall x, (x \in B \iff x \subseteq a).$$

\begin{axiom}

  \lean*{Mathlib/Init/Set}{Set.powerset}

\end{axiom}

\section{\defined{Subset Axioms}}%
\label{ref:subset-axioms}

For each formula $\phi$ not containing $B$, the following is an axiom:
  $$\forall t_1, \cdots \forall t_k, \forall c,
      \exists B, \forall x, (x \in B \iff x \in c \land \phi).$$

\begin{axiom}

  \lean*{Mathlib/Init/Set}{Set.Subset}

\end{axiom}

\section{\defined{Union Axiom}}%
\label{ref:union-axiom}

For any set $A$, there exists a set $B$ whose elements are exactly the members
  of the members of $A$:
  $$\forall A, \exists B, \forall x \left[ x \in B \iff (\exists b \in A) x \in b \right]$$

\begin{axiom}

  \lean*{Mathlib/Data/Set/Lattice}{Set.sUnion}

\end{axiom}

\section{\defined{Union Axiom, Preliminary Form}}%
\label{ref:union-axiom-preliminary-form}

For any sets $a$ and $b$, there is a set whose members are those sets belonging
  either to $a$ or to $b$ (or both):
  $$\forall a, \forall b, \exists B, \forall x,
      (x \in B \iff x \in a \text{ or } x \in b).$$

\begin{axiom}

  \lean*{Mathlib/Init/Set}{Set.union}

\end{axiom}

\endgroup

\chapter{Introduction}%
\label{chap:introduction}

\section{Baby Set Theory}%
\label{sec:baby-set-theory}

\subsection{\verified{Exercise 1.1}}%
\label{sub:exercise-1.1}

Which of the following become true when "$\in$" is inserted in place of the
  blank?
Which become true when "$\subseteq$" is inserted?

\subsubsection{\verified{Exercise 1.1a}}%
\label{ssub:exercise-1.1a}

$\{\emptyset\} \_\_\_\_ \{\emptyset, \{\emptyset\}\}$.

\begin{proof}

  \lean{Bookshelf/Enderton/Set/Chapter\_1}
    {Enderton.Set.Chapter\_1.exercise\_1\_1a}

  Because the \textit{object} $\{\emptyset\}$ is a member of the right-hand set,
    the statement is \textbf{true} in the case of "$\in$".

  Because the \textit{members} of $\{\emptyset\}$ are all members of the
    right-hand set, the statement is also \textbf{true} in the case of
    "$\subseteq$".

\end{proof}

\subsubsection{\verified{Exercise 1.1b}}%
\label{ssub:exercise-1.11b}

$\{\emptyset\} \_\_\_\_ \{\emptyset, \{\{\emptyset\}\}\}$.

\begin{proof}

  \lean{Bookshelf/Enderton/Set/Chapter\_1}
    {Enderton.Set.Chapter\_1.exercise\_1\_1b}

  Because the \textit{object} $\{\emptyset\}$ is not a member of the right-hand
    set, the statement is \textbf{false} in the case of "$\in$".

  Because the \textit{members} of $\{\emptyset\}$ are all members of the
    right-hand set, the statement is \textbf{true} in the case of "$\subseteq$".

\end{proof}

\subsubsection{\verified{Exercise 1.1c}}%
\label{ssub:exercise-1.1c}

$\{\{\emptyset\}\} \_\_\_\_ \{\emptyset, \{\emptyset\}\}$.

\begin{proof}

  \lean{Bookshelf/Enderton/Set/Chapter\_1}
    {Enderton.Set.Chapter\_1.exercise\_1\_1c}

  Because the \textit{object} $\{\{\emptyset\}\}$ is not a member of the
    right-hand set, the statement is \textbf{false} in the case of "$\in$".

  Because the \textit{members} of $\{\{\emptyset\}\}$ are all members of the
    right-hand set, the statement is \textbf{true} in the case of "$\subseteq$".

\end{proof}

\subsubsection{\verified{Exercise 1.1d}}%
\label{ssub:exercise-1.1d}

$\{\{\emptyset\}\} \_\_\_\_ \{\emptyset, \{\{\emptyset\}\}\}$.

\begin{proof}

  \lean{Bookshelf/Enderton/Set/Chapter\_1}
    {Enderton.Set.Chapter\_1.exercise\_1\_1d}

  Because the \textit{object} $\{\{\emptyset\}\}$ is a member of the right-hand
    set, the statement is \textbf{true} in the case of "$\in$".

  Because the \textit{members} of $\{\{\emptyset\}\}$ are not all members of the
    right-hand set, the statement is \textbf{false} in the case of
    "$\subseteq$".

\end{proof}

\subsubsection{\verified{Exercise 1.1e}}%
\label{ssub:exercise-1.1e}

$\{\{\emptyset\}\} \_\_ \{\emptyset, \{\emptyset, \{\emptyset\}\}\}$.

\begin{proof}

  \lean{Bookshelf/Enderton/Set/Chapter\_1}
    {Enderton.Set.Chapter\_1.exercise\_1\_1e}

  Because the \textit{object} $\{\{\emptyset\}\}$ is not a member of the
    right-hand set, the statement is \textbf{false} in the case of "$\in$".

  Because the \textit{members} of $\{\{\emptyset\}\}$ are not all members of the
    right-hand set, the statement is \textbf{false} in the case of
    "$\subseteq$".

\end{proof}

\subsection{\verified{Exercise 1.2}}%
\label{sub:exercise-1.2}

Show that no two of the three sets $\emptyset$, $\{\emptyset\}$, and
  $\{\{\emptyset\}\}$ are equal to each other.

\begin{proof}

  \lean{Bookshelf/Enderton/Set/Chapter\_1}
    {Enderton.Set.Chapter\_1.exercise\_1\_2}

  By the \nameref{ref:extensionality-axiom}, $\emptyset$ is only equal to
    $\emptyset$.
  This immediately shows it is not equal to the other two.
  Now consider object $\emptyset$.
  This object is a member of $\{\emptyset\}$ but is not a member of
    $\{\{\emptyset\}\}$.
  Again, by the \nameref{ref:extensionality-axiom}, these two sets must be
    different.

\end{proof}

\subsection{\verified{Exercise 1.3}}%
\label{sub:exercise-1.3}

Show that if $B \subseteq C$, then $\powerset{B} \subseteq \powerset{C}$.

\begin{proof}

  \lean{Bookshelf/Enderton/Set/Chapter\_1}
    {Enderton.Set.Chapter\_1.exercise\_1\_3}

  Let $x \in \powerset{B}$.
  By definition of the \nameref{ref:power-set}, $x$ is a subset of $B$.
  By hypothesis, $B \subseteq C$.
  Then $x \subseteq C$.
  Again by definition of the \nameref{ref:power-set}, it follows
    $x \in \powerset{C}$.

\end{proof}

\subsection{\verified{Exercise 1.4}}%
\label{sub:exercise-1.4}

Assume that $x$ and $y$ are members of a set $B$.
Show that $\{\{x\}, \{x, y\}\} \in \powerset{\powerset{B}}.$

\begin{proof}

  \lean{Bookshelf/Enderton/Set/Chapter\_1}
    {Enderton.Set.Chapter\_1.exercise\_1\_4}

  Let $x$ and $y$ be members of set $B$.
  Then $\{x\}$ and $\{x, y\}$ are subsets of $B$.
  By definition of the \nameref{ref:power-set}, $\{x\}$ and $\{x, y\}$ are
    members of $\powerset{B}$.
  Then $\{\{x\}, \{x, y\}\}$ is a subset of $\powerset{B}$.
  By definition of the \nameref{ref:power-set}, $\{\{x\}, \{x, y\}\}$ is a
    member of $\powerset{\powerset{B}}$.

\end{proof}

\section{Sets - An Informal View}%
\label{sec:sets-informal-view}

\subsection{\partial{Exercise 2.1}}%
\label{sub:exercise-2.1}

Define the rank of a set $c$ to be the least $\alpha$ such that
  $c \subseteq V_\alpha$.
Compute the rank of $\{\{\emptyset\}\}$.
Compute the rank of
  $\{\emptyset, \{\emptyset\}, \{\emptyset, \{\emptyset\}\}\}$.

\begin{proof}

  We first compute the values of $V_n$ for $0 \leq n \leq 3$ under the
    assumption the set of atoms $A$ at the bottom of the hierarchy is empty.
  \begin{align*}
    V_0 & = \emptyset \\
    V_1 & = V_0 \cup \powerset{V_0} \\
        & = \emptyset \cup \{\emptyset\} \\
        & = \{\emptyset\} \\
    V_2 & = V_1 \cup \powerset{V_1} \\
        & = \{\emptyset\} \cup \powerset{\{\emptyset\}} \\
        & = \{\emptyset\} \cup \{\emptyset, \{\emptyset\}\} \\
        & = \{\emptyset, \{\emptyset\}\} \\
    V_3 & = V_2 \cup \powerset{V_2} \\
        & = \{\emptyset, \{\emptyset\}\} \cup
            \powerset{\{\emptyset, \{\emptyset\}\}} \\
        & = \{\emptyset, \{\emptyset\}\} \cup
            \{\emptyset,
              \{\emptyset\},
              \{\{\emptyset\}\},
              \{\emptyset, \{\emptyset\}\}\} \\
        & = \{\emptyset,
              \{\emptyset\},
              \{\{\emptyset\}\},
              \{\emptyset, \{\emptyset\}\}\}
  \end{align*}
  It then immediately follows $\{\{\emptyset\}\}$ has rank $2$ and
    $\{\emptyset, \{\emptyset\}, \{\emptyset, \{\emptyset\}\}\}$ has rank $3$.

\end{proof}

\subsection{\partial{Exercise 2.2}}%
\label{sub:exercise-2.2}

We have stated that $V_{\alpha + 1} = A \cup \powerset{V_\alpha}$.
Prove this at least for $\alpha < 3$.

\begin{proof}

  Let $A$ be the set of atoms in our set hierarchy.
  Let $P(n)$ be the predicate, "$V_{n + 1} = A \cup \powerset{V_n}$."
  We prove $P(n)$ holds true for all natural numbers $n \geq 1$ via induction.

  \paragraph{Base Case}%

    Let $n = 1$.
    By definition, $V_1 = V_0 \cup \powerset{V_0}$.
    By definition, $V_0 = A$.
    Therefore $V_1 = A \cup \powerset{V_0}$.
    This proves $P(1)$ holds true.

  \paragraph{Induction Step}%

    Suppose $P(n)$ holds true for some $n \geq 1$.
    Consider $V_{n+1}$.
    By definition, $V_{n+1} = V_n \cup \powerset{V_n}$.
    Therefore, by the induction hypothesis,
      \begin{align}
        V_{n+1}
          & = V_n \cup \powerset{V_n}
            \nonumber \\
          & = (A \cup \powerset{V_{n-1}}) \cup \powerset{V_n}
            \nonumber \\
          & = A \cup (\powerset{V_{n-1}} \cup \powerset{V_n})
            \label{sub:exercise-2.2-eq1}
      \end{align}
    But $V_{n-1}$ is a subset of $V_n$.
    \nameref{sub:exercise-1.3} then implies
      $\powerset{V_{n-1}} \subseteq \powerset{V_n}$.
    This means \eqref{sub:exercise-2.2-eq1} can be simplified to
      $$V_{n+1} = A \cup \powerset{V_n},$$
    proving $P(n+1)$ holds true.

  \paragraph{Conclusion}%

    By mathematical induction, it follows for all $n \geq 1$, $P(n)$ is true.

\end{proof}

\subsection{\partial{Exercise 2.3}}%
\label{sub:exercise-2.3}

List all the members of $V_3$.
List all the members of $V_4$.
(It is to be assumed here that there are no atoms.)

\begin{proof}

  As seen in the proof of \nameref{sub:exercise-2.1},
    $$V_3 = \{
        \emptyset,
        \{\emptyset\},
        \{\{\emptyset\}\},
        \{\emptyset, \{\emptyset\}\}
    \}.$$
  By \nameref{sub:exercise-2.2}, $V_4 = \powerset{V_3}$ (since it is assumed
    there are no atoms).
  Thus
    \begin{align*}
      & V_4 = \{ \\
      & \qquad \emptyset, \\
      & \qquad \{\emptyset\}, \\
      & \qquad \{\{\emptyset\}\}, \\
      & \qquad \{\{\{\emptyset\}\}\}, \\
      & \qquad \{\{\emptyset, \{\emptyset\}\}\}, \\
      & \qquad \{\emptyset, \{\emptyset\}\}, \\
      & \qquad \{\emptyset, \{\{\emptyset\}\}\}, \\
      & \qquad \{\emptyset, \{\emptyset, \{\emptyset\}\}\}, \\
      & \qquad \{\{\emptyset\}, \{\{\emptyset\}\}\}, \\
      & \qquad \{\{\emptyset\}, \{\emptyset, \{\emptyset\}\}\}, \\
      & \qquad \{\{\{\emptyset\}\}, \{\emptyset, \{\emptyset\}\}\}, \\
      & \qquad \{\emptyset, \{\emptyset\}, \{\{\emptyset\}\}\}, \\
      & \qquad \{\emptyset, \{\emptyset\}, \{\emptyset, \{\emptyset\}\}\}, \\
      & \qquad \{\emptyset, \{\{\emptyset\}\}, \{\emptyset, \{\emptyset\}\}\} \\
      & \qquad \{\{\emptyset\}, \{\{\emptyset\}\}, \{\emptyset, \{\emptyset\}\}\}, \\
      & \qquad \{\emptyset, \{\emptyset\}, \{\{\emptyset\}\}, \{\emptyset, \{\emptyset\}\}\} \\
      & \}.
    \end{align*}

\end{proof}

\chapter{Axioms and Operations}%
\label{chap:axioms-operations}

\section{Axioms}%
\label{sec:axioms}

\subsection{\unverified{Theorem 2A}}%
\label{sub:theorem-2a}

\begin{theorem}[2A]

  There is no set to which every set belongs.

\end{theorem}

\begin{proof}

  TODO

\end{proof}

\subsection{\unverified{Theorem 2B}}%
\label{sub:theorem-2b}

\begin{theorem}[2B]

  For any nonempty set $A$, there exists a unique set $B$ such that for any
    $x$, $$x \in B \iff x \text{ belongs to every member of } A.$$

\end{theorem}

\begin{proof}

  TODO

\end{proof}

\section{Exercises 3}%
\label{sec:exercises-3}

\subsection{\verified{Exercise 3.1}}%
\label{sub:exercise-3.1}

Assume that $A$ is the set of integers divisible by $4$.
Similarly assume that $B$ and $C$ are the sets of integers divisible by $9$ and
  $10$, respectively.
What is in $A \cap B \cap C$?

\begin{answer}

  \lean{Bookshelf/Enderton/Set/Chapter\_1}
    {Enderton.Set.Chapter\_1.exercise\_3\_1}

  The set of integers divisible by $4$, $9$, and $10$.

\end{answer}

\subsection{\verified{Exercise 3.2}}%
\label{sub:exercise-3.2}

Give an example of sets $A$ and $B$ for which $\bigcup A = \bigcup B$ but
  $A \neq B$.

\begin{answer}

  \lean{Bookshelf/Enderton/Set/Chapter\_1}
    {Enderton.Set.Chapter\_1.exercise\_3\_2}

  Let $A = \{\{1\}, \{2\}\}$ and $B = \{\{1, 2\}\}$.

\end{answer}

\subsection{\verified{Exercise 3.3}}%
\label{sub:exercise-3.3}

Show that every member of a set $A$ is a subset of $\bigcup A$.
(This was stated as an example in this section.)

\begin{proof}

  \lean{Bookshelf/Enderton/Set/Chapter\_1}
    {Enderton.Set.Chapter\_1.exercise\_3\_3}

  Let $x \in A$.
  By definition, $$\bigcup A = \{ y \mid (\exists b \in A) y \in b\}.$$
  Then $\{ y \mid y \in x\} \subseteq \bigcup A$.
  But $\{ y \mid y \in x\} = x$.
  Thus $x \subseteq \bigcup A$.

\end{proof}

\subsection{\verified{Exercise 3.4}}%
\label{sub:exercise-3.4}

Show that if $A \subseteq B$, then $\bigcup A \subseteq \bigcup B$.

\begin{proof}

  \lean{Bookshelf/Enderton/Set/Chapter\_1}
    {Enderton.Set.Chapter\_1.exercise\_3\_4}

  Let $A$ and $B$ be sets such that $A \subseteq B$.
  Let $x \in \bigcup A$.
  By definition of the union, there exists some $b \in A$ such that $x \in b$.
  By definition of the subset, $b \in B$.
  This immediatley implies $x \in \bigcup B$.
  Since this holds for all $x \in \bigcup A$, it follows
    $\bigcup A \subseteq \bigcup B$.

\end{proof}

\subsection{\verified{Exercise 3.5}}%
\label{sub:exercise-3.5}

Assume that every member of $\mathscr{A}$ is a subset of $B$.
Show that $\bigcup \mathscr{A} \subseteq B$.

\begin{proof}

  \lean{Bookshelf/Enderton/Set/Chapter\_1}
    {Enderton.Set.Chapter\_1.exercise\_3\_5}

  Let $x \in \bigcup \mathscr{A}$.
  By definition,
    $$\bigcup \mathscr{A} = \{ y \mid (\exists b \in A)y \in b \}.$$
  Then there exists some $b \in A$ such that $x \in b$.
  By hypothesis, $b \subseteq B$.
  Thus $x$ must also be a member of $B$.
  Since this holds for all $x \in \bigcup \mathscr{A}$, it follows
    $\bigcup \mathscr{A} \subseteq B$.

\end{proof}

\subsection{\verified{Exercise 3.6a}}%
\label{sub:exercise-3.6a}

Show that for any set $A$, $\bigcup \powerset{A} = A$.

\begin{proof}

  \lean{Bookshelf/Enderton/Set/Chapter\_1}
    {Enderton.Set.Chapter\_1.exercise\_3\_6a}

  We prove that (i) $\bigcup \powerset{A} \subseteq A$ and (ii)
    $A \subseteq \bigcup \powerset{A}$.

  \paragraph{(i)}%
  \label{par:exercise-3.6a-i}

    By definition, the \nameref{ref:power-set} of $A$ is the set of all subsets
      of $A$.
    In other words, every member of $\powerset{A}$ is a subset of $A$.
    By \nameref{sub:exercise-3.5}, $\bigcup \powerset{A} \subseteq A$.

  \paragraph{(ii)}%
  \label{par:exercise-3.6a-ii}

    Let $x \in A$.
    By definition of the power set of $A$, $\{x\} \in \powerset{A}$. 
    By definition of the union,
      $$\bigcup \powerset{A} =
        \{ y \mid (\exists b \in \powerset{A}), y \in b).$$
    Since $x \in \{x\}$ and $\{x\} \in \powerset{A}$, it follows
      $x \in \bigcup \powerset{A}$.
    Thus $A \subseteq \bigcup \powerset{A}$.

  \paragraph{Conclusion}%

    By \nameref{par:exercise-3.6a-i} and \nameref{par:exercise-3.6a-ii},
      $\bigcup \powerset{A} = A$.

\end{proof}

\subsection{\verified{Exercise 3.6b}}%
\label{sub:exercise-3.6b}

Show that $A \subseteq \powerset{\bigcup A}$.
Under what conditions does equality hold?

\begin{proof}

  \lean{Bookshelf/Enderton/Set/Chapter\_1}
    {Enderton.Set.Chapter\_1.exercise\_3\_6b}

  Let $x \in A$.
  By \nameref{sub:exercise-3.3}, $x$ is a subset of $\bigcup A$.
  By the definition of the \nameref{ref:power-set},
    $$\powerset{\bigcup A} = \{ y \mid y \subseteq \bigcup A \}.$$
  Therefore $x \in \powerset{\bigcup A}$.
  Since this holds for all $x \in A$, $A \subseteq \powerset{\bigcup A}$.

  \suitdivider

  We show equality holds if and only if there exists some set $B$ such that
    $A = \powerset{B}$.

  \paragraph{($\Rightarrow$)}%
  \label{par:exercise-3.6b-right}

    Suppose $A = \powerset{\bigcup A}$.
    Then our statement immediately follows by settings $B = \bigcup A$.

  \paragraph{($\Leftarrow$)}%
  \label{par:exercise-3.6b-left}

    Suppose there exists some set $B$ such that $A = \powerset{B}$.
    Therefore
      \begin{align*}
        \powerset{\bigcup A}
          & = \powerset{\left(\bigcup {\powerset {B}}\right)} \\
          & = \powerset{B} & \textref{sub:exercise-3.6a} \\
          & = A.
      \end{align*}

  \paragraph{Conclusion}%

    By \nameref{par:exercise-3.6b-right} and \nameref{par:exercise-3.6b-left},
      $A = \powerset{\bigcup A}$ if and only if there exists some set $B$ such
      that $A = \powerset{B}$.

\end{proof}

\subsection{\verified{Exercise 3.7a}}%
\label{sub:exercise-3.7a}

Show that for any sets $A$ and $B$,
  $$\powerset{A} \cap \powerset{B} = \powerset{(A \cap B)}.$$

\begin{proof}

  \lean{Bookshelf/Enderton/Set/Chapter\_1}
    {Enderton.Set.Chapter\_1.exercise\_3\_7a}

  Let $A$ and $B$ be arbitrary sets. We show that
    $\powerset{A} \cap \powerset{B} \subseteq \powerset{(A \cap B)}$ and then
    show that $\powerset{A} \cap \powerset{B} \supseteq \powerset{(A \cap B)}$.

  \paragraph{($\subseteq$)}%

    Let $x \in \powerset{A} \cap \powerset{B}$.
    That is, $x \in \powerset{A}$ and $x \in \powerset{B}$.
    By the definition of the \nameref{ref:power-set},
      \begin{align*}
        \powerset{A} & = \{ y \mid y \subseteq A \} \\
        \powerset{B} & = \{ y \mid y \subseteq B \}
      \end{align*}
    Thus $x \subseteq A$ and $x \subseteq B$, meaning $x \subseteq A \cap B$.
    But then $x \in \powerset{(A \cap B)}$, the set of all subsets of
      $A \cap B$.
    Since this holds for all $x \in \powerset{A} \cap \powerset{B}$, it follows
      $$\powerset{A} \cap \powerset{B} \subseteq \powerset{(A \cap B)}.$$

  \paragraph{($\supseteq$)}%

    Let $x \in \powerset{(A \cap B)}$.
    By the definition of the \nameref{ref:power-set},
      $$\powerset{(A \cap B)} = \{ y \mid y \subseteq A \cap B \}.$$
    Thus $x \subseteq A \cap B$, meaning $x \subseteq A$ and $x \subseteq B$.
    But this implies $x \in \powerset{A}$, the set of all subsets of $A$.
    Likewise $x \in \powerset{B}$, the set of all subsets of $B$.
    Thus $x \in \powerset{A} \cap \powerset{B}$.
    Since this holds for all $x \in \powerset{(A \cap B)}$, it follows
      $$\powerset{(A \cap B)} \subseteq \powerset{A} \cap \powerset{B}.$$

  \paragraph{Conclusion}%

    Since each side of our identity is a subset of the other,
      $$\powerset{(A \cap B)} = \powerset{A} \cap \powerset{B}.$$

\end{proof}

\subsection{\verified{Exercise 3.7b}}%
\label{sub:exercise-3.7b}

Show that $\powerset{A} \cup \powerset{B} \subseteq \powerset{(A \cup B)}$.
Under what conditions does equality hold?

\begin{proof}

  \  % Add space.

  \lean*{Bookshelf/Enderton/Set/Chapter\_1}
    {Enderton.Set.Chapter\_1.exercise\_3\_7b\_i}

  \lean{Bookshelf/Enderton/Set/Chapter\_1}
    {Enderton.Set.Chapter\_1.exercise\_3\_7b\_ii}

  Let $x \in \powerset{A} \cup \powerset{B}$.
  By definition, $x \in \powerset{A}$ or $x \in \powerset{B}$ (or both).
  By the definition of the \nameref{ref:power-set},
    \begin{align*}
      \powerset{A} &= \{ y \mid y \subseteq A \} \\
      \powerset{B} &= \{ y \mid y \subseteq B \}.
    \end{align*}
  Thus $x \subseteq A$ or $x \subseteq B$.
  Therefore $x \subseteq A \cup B$.
  But then $x \in \powerset{(A \cup B)}$, the set of all subsets of $A \cup B$.

  \suitdivider

  We show equality holds if and only if one of $A$ or $B$ is a subset of the
    other.

  \paragraph{($\Rightarrow$)}%
  \label{par:exercise-3.7b-right}

    Suppose
      \begin{equation}
        \label{sub:exercise-3.7b-eq1}
        \powerset{A} \cup \powerset{B} = \powerset{(A \cup B)}.
      \end{equation}
    By the definition of the \nameref{ref:power-set},
      $A \cup B \in \powerset{(A \cup B)}$.
    Then \eqref{sub:exercise-3.7b-eq1} implies
      $A \cup B \in \powerset{A} \cup \powerset{B}$.
    That is, $A \cup B \in \powerset{A}$ or $A \cup B \in \powerset{B}$ (or
      both).

    For the sake of contradiction, suppose $A \not\subseteq B$ and
      $B \not\subseteq A$.
    Then there exists an element $x \in A$ such that $x \not\in B$ and there
      exists an element $y \in B$ such that $y \not\in A$.
    But then $A \cup B \not\in \powerset{A}$ since $y$ cannot be a member of a
      member of $\powerset{A}$.
    Likewise, $A \cup B \not\in \powerset{B}$ since $x$ cannot be a member of a
      member of $\powerset{B}$.
    Therefore our assumption is incorrect.
    In other words, $A \subseteq B$ or $B \subseteq A$.

  \paragraph{($\Leftarrow$)}%
  \label{par:exercise-3.7b-left}

    WLOG, suppose $A \subseteq B$.
    Then, by \nameref{sub:exercise-1.3}, $\powerset{A} \subseteq \powerset{B}$.
    Thus
      \begin{align*}
        \powerset{A} \cup \powerset{B}
          & = \powerset{B} \\
          & = \powerset{A \cup B}.
      \end{align*}

  \paragraph{Conclusion}%

    By \nameref{par:exercise-3.7b-right} and \nameref{par:exercise-3.7b-left},
      it follows
      $\powerset{A} \cup \powerset{B} \subseteq \powerset{(A \cup B)}$ if and
      only if $A \subseteq B$ or $B \subseteq A$.

\end{proof}

\subsection{\partial{Exercise 3.8}}%
\label{sub:exercise-3.8}

Show that there is no set to which every singleton (that is, every set of the
  form $\{x\}$) belongs.
[\textit{Suggestion}: Show that from such a set, we could construct a set to
  which every set belonged.]

\begin{proof}

  We proceed by contradiction.
  Suppose there existed a set $A$ consisting of every singleton.
  Then the \nameref{ref:union-axiom} suggests $\bigcup A$ is a set.
  But this set is precisely the class of all sets, which is \textit{not} a set.
  Thus our original assumption was incorrect.
  That is, there is no set to which every singleton belongs.

\end{proof}

\subsection{\verified{Exercise 3.9}}%
\label{sub:exercise-3.9}

Give an example of sets $a$ and $B$ for which $a \in B$ but
  $\powerset{a} \not\in \powerset{B}$.

\begin{answer}

  \lean{Bookshelf/Enderton/Set/Chapter\_1}
    {Enderton.Set.Chapter\_1.exercise\_3\_9}

  Let $a = \{1\}$ and $B = \{\{1\}\}$.
  Then
    \begin{align*}
      \powerset{a} & = \{\emptyset, \{1\}\} \\
      \powerset{B} & = \{\emptyset, \{\{1\}\}\}.
    \end{align*}
  It immediately follows that $\powerset{a} \not\in \powerset{B}$.

\end{answer}

\subsection{\verified{Exercise 3.10}}%
\label{sub:exercise-3.10}

Show that if $a \in B$, then $\powerset{a} \in \powerset{\powerset{\bigcup B}}$.
[\textit{Suggestion}: If you need help, look in the Appendix.]

\begin{proof}

  \lean{Bookshelf/Enderton/Set/Chapter\_1}
    {Enderton.Set.Chapter\_1.exercise\_3\_10}

  Suppose $a \in B$.
  By \nameref{sub:exercise-3.3}, $a \subseteq \bigcup B$.
  By \nameref{sub:exercise-1.3}, $\powerset{a} \subseteq \powerset{\bigcup B}$.
  By the definition of the \nameref{ref:power-set},
    $$\powerset{\powerset{\bigcup B}} =
      \{ y \mid y \subseteq \powerset{\bigcup B} \}.$$
  Therefore $\powerset{a} \in \powerset{\powerset{\bigcup B}}$.

\end{proof}

\end{document}
