\usepackage{amsfonts, amsmath, amssymb, amsthm}
\usepackage{bigfoot}
\usepackage{comment}
\usepackage[shortlabels]{enumitem}
\usepackage{etoolbox}
\usepackage{environ}
\usepackage{fancybox}
\usepackage{fontawesome5}
\usepackage{mathabx, mathrsfs}
\usepackage{soul}
\usepackage{stmaryrd}
% Must load `xcolor` before `tikz`.
\usepackage[dvipsnames]{xcolor}
\usepackage{tikz}
% `hyperref` comes after `xr-hyper`.
\usepackage{xr-hyper}
\usepackage{hyperref}

% Open "private" namespace.
\makeatletter

% ========================================
% General
% ========================================

\newcommand{\header}[2]{\title{#1}\author{#2}\date{}\maketitle}

% ========================================
% Dividers
% ========================================

\newcommand\@linespace{\vspace{10pt}}
\newcommand\linedivider{\@linespace\hrule\@linespace}
\WithSuffix\newcommand\linedivider*{\@linespace\hrule}
\newcommand\suitdivider{$$\spadesuit\;\spadesuit\;\spadesuit$$}

% ========================================
% Linking
% ========================================

\hypersetup{colorlinks=true, linkcolor=blue, urlcolor=blue}
\newcommand{\textref}[1]{\text{\nameref{#1}}}
\newcommand{\hyperlabel}[1]{%
  \label{#1}%
  \hypertarget{#1}{}}

% Denote whether we are working with a standard/Mathlib statement (lean) or a
% custom one (code).
\newcommand\@leanlink[4]{%
  \textcolor{blue}{\raisebox{-4.5pt}{%
    \tikz{\draw (0, 0) node[yscale=-1,xscale=1] {\faFont};}}}%
  {-\;}\href{#1/#2.html\##3}{#4}}

\newcommand\@codelink[4]{%
  \textcolor{blue}{\raisebox{-4.5pt}{%
    \tikz{\draw (0, 0) node[] {\faCodeBranch};}}}%
  {-\;}\href{#1/#2.html\##3}{#4}}

% Reference to an anchor of Lean documentation.
\newcommand\leanref[3]{%
  \@leanlink{#1}{#2}{#3}{#3}\vspace{10pt}}
\WithSuffix\newcommand\leanref*[3]{%
  \@leanlink{#1}{#2}{#3}{#3}}

\newcommand\coderef[3]{%
  \@codelink{#1}{#2}{#3}{#3}\vspace{10pt}}
\WithSuffix\newcommand\coderef*[3]{%
  \@codelink{#1}{#2}{#3}{#3}}

% Variants that allows customizing display text.
\newcommand\leanpref[4]{%
  \@leanlink{#1}{#2}{#3}{#4}\vspace{10pt}}
\WithSuffix\newcommand\leanpref*[4]{%
  \@leanlink{#1}{#2}{#3}{#4}}

\newcommand\codepref[4]{%
  \@codelink{#1}{#2}{#3}{#4}\vspace{10pt}}
\WithSuffix\newcommand\codepref*[4]{%
  \@codelink{#1}{#2}{#3}{#4}}

% Macro to build all Lean related commands relative to a specified directory.
\newcommand\makeleancommands[1]{%
  \newcommand\lean[2]{%
    \leanref{#1}{##1}{##2}}
  \WithSuffix\newcommand\lean*[2]{%
    \leanref*{#1}{##1}{##2}}

  \newcommand\code[2]{%
    \coderef{#1}{##1}{##2}}
  \WithSuffix\newcommand\code*[2]{%
    \coderef*{#1}{##1}{##2}}

  \newcommand\leanp[3]{%
    \leanpref{#1}{##1}{##2}{##3}}
  \WithSuffix\newcommand\leanp*[3]{%
    \leanpref*{#1}{##1}{##2}{##3}}

  \newcommand\codep[3]{%
    \codepref{#1}{##1}{##2}{##3}}
  \WithSuffix\newcommand\codep*[3]{%
    \codepref*{#1}{##1}{##2}{##3}}
}

% ========================================
% Admonitions
% ========================================

\newcommand{\@admonition}[2]{%
  \begin{center}
    \doublebox{
      \begin{minipage}{0.95\textwidth}
        \vspace{2pt}
        \hl{#1} #2
        \vspace{2pt}
      \end{minipage}}
  \end{center}}

\NewEnviron{note}{\@admonition{NOTE:}{\BODY}}
\NewEnviron{todo}{\@admonition{TODO:}{\BODY}}

% ========================================
% Statements
% ========================================

\newcommand\@statement[1]{%
  \linedivider*\paragraph{\normalfont\normalsize\textit{#1.}}}
\newcommand{\statementpadding}{\ \vspace{8pt}}

\newenvironment{answer}{\@statement{Answer}}{\hfill$\square$}
\newenvironment{axiom}{\@statement{Axiom}}{\hfill$\square$}
\newenvironment{definition}{\@statement{Definition}}{\hfill$\square$}
\renewenvironment{proof}{\@statement{Proof}}{\hfill$\square$}

\newtheorem{corollaryinner}{Corollary}
\newenvironment{corollary}[1][]{%
  \ifstrempty{#1}
    {\corollaryinner}
    {\renewcommand\thecorollaryinner{#1}\corollaryinner}
}{\endcorollaryinner}

\newtheorem{lemmainner}{Lemma}
\newenvironment{lemma}[1][]{%
  \ifstrempty{#1}
    {\lemmainner}
    {\renewcommand\thelemmainner{#1}\lemmainner}
}{\endlemmainner}

\newtheorem{theoreminner}{Theorem}
\newenvironment{theorem}[1][]{%
  \ifstrempty{#1}
    {\theoreminner}
    {\renewcommand\thetheoreminner{#1}\theoreminner}
}{\endtheoreminner}

% ========================================
% Status
% ========================================

\DeclareRobustCommand{\defined}[1]{%
  \texorpdfstring{\color{darkgray}\faParagraph\ #1}{#1}}
\DeclareRobustCommand{\verified}[1]{%
  \texorpdfstring{\color{teal}\faCheckCircle\ #1}{#1}}
\DeclareRobustCommand{\unverified}[1]{%
  \texorpdfstring{\color{olive}\faCheckCircle[regular]\ #1}{#1}}
\DeclareRobustCommand{\pending}[1]{%
  \texorpdfstring{\color{Fuchsia}\faPencil*\ #1}{#1}}
\DeclareRobustCommand{\sorry}[1]{%
  \texorpdfstring{\color{Maroon}\faExclamationCircle\ #1}{#1}}

% ========================================
% Math
% ========================================

\newcommand{\abs}[1]{\left|#1\right|}
\newcommand{\ceil}[1]{\left\lceil#1\right\rceil}
\newcommand{\dom}[1]{\textop{dom}{#1}}
\newcommand{\fld}[1]{\textop{fld}{#1}}
\newcommand{\floor}[1]{\left\lfloor#1\right\rfloor}
\newcommand{\icc}[2]{\left[#1, #2\right]}
\newcommand{\ico}[2]{\left[#1, #2\right)}
\newcommand{\img}[2]{#1\!\left\llbracket#2\right\rrbracket}
\newcommand{\ioc}[2]{\left(#1, #2\right]}
\newcommand{\ioo}[2]{\left(#1, #2\right)}
\newcommand{\powerset}[1]{\mathscr{P}#1}
\newcommand{\ran}[1]{\textop{ran}{#1}}
\newcommand{\textop}[1]{\mathop{\text{#1}}}
\newcommand{\ubar}[1]{\text{\b{$#1$}}}

\let\oldemptyset\emptyset
\let\emptyset\varnothing

% Close off "private" namespace.
\makeatother
