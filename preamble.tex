\usepackage{amsfonts, amsmath, amsthm}
\usepackage[shortlabels]{enumitem}
\usepackage{environ}
\usepackage{fancybox}
\usepackage{fontawesome5}
\usepackage{mathrsfs}
\usepackage{soul}
\usepackage{xcolor}
% `hyperref` comes after `xr-hyper`.
\usepackage{xr-hyper}
\usepackage{hyperref}

% ========================================
% Linking
% ========================================

\hypersetup{colorlinks=true, urlcolor=blue}
\newcommand{\leanref}[2]{{\color{blue}$\pmb{\exists}\;{-}\;$}\href{#1}{#2}}

% ========================================
% Environments
% ========================================

\newenvironment{axiom}{%
  \paragraph{\normalfont\normalsize\textit{Axiom.}}}
  {\hfill$\square$}
\newcommand{\divider}{%
  \vspace{10pt}
  \hrule
  \vspace{10pt}}
\newcommand{\header}[2]{%
  \title{#1}
  \author{#2}
  \date{}
  \maketitle}
\newcommand{\note}[1]{%
  \begin{center}
    \doublebox{
      \begin{minipage}{0.95\textwidth}
        \vspace{2pt}
        \hl{Note:} #1
        \vspace{2pt}
      \end{minipage}}
  \end{center}}

% ========================================
% Status
% ========================================

% Used for statements encoded in both LaTeX and Lean.
\DeclareRobustCommand{\defined}[1]{%
  \texorpdfstring{\color{teal}\faParagraph\ #1}{#1}}

% Used for statements proven in both LaTeX and Lean.
\DeclareRobustCommand{\verified}[1]{%
  \texorpdfstring{\color{teal}\faCheckCircle\ #1}{#1}}

% Used for statements proven or encoded in LaTeX but not yet in Lean.
\DeclareRobustCommand{\partial}[1]{%
  \texorpdfstring{\color{magenta}\faPencil*\ #1}{#1}}

% A catch-all for anything that doesn't fit the above categories. Incomplete
% definitions, statements without proof, etc. belong here.
\DeclareRobustCommand{\unverified}[1]{%
  \texorpdfstring{\color{red}\faExclamationCircle\ #1}{#1}}

% ========================================
% Math
% ========================================

\newcommand{\abs}[1]{\left|#1\right|}
\newcommand{\ceil}[1]{\left\lceil#1\right\rceil}
\newcommand{\floor}[1]{\left\lfloor#1\right\rfloor}
\newcommand{\icc}[2]{\left[#1, #2\right]}
\newcommand{\ico}[2]{\left[#1, #2\right)}
\newcommand{\ioc}[2]{\left(#1, #2\right]}
\newcommand{\ioo}[2]{\left(#1, #2\right)}
