\documentclass{article}
\usepackage[shortlabels]{enumitem}

\usepackage{amsfonts, amsthm}
\usepackage{hyperref}

\newtheorem{theorem}{Theorem}
\newtheorem{xtheoreminner}{Theorem}
\newenvironment{xtheorem}[1]{%
  \renewcommand\thextheoreminner{#1}%
  \xtheoreminner
}{\endxtheoreminner}

\hypersetup{colorlinks=true, urlcolor=blue}

\newcommand{\ceil}[1]{\left\lceil#1\right\rceil}
\newcommand{\floor}[1]{\left\lfloor#1\right\rfloor}
% The first argument refers to a relative path upward from a current file to
% the root of the workspace (i.e. where this `preamble.tex` file is located).
\newcommand{\lean}[4]{\href{#1/#2.html\##3}{#4}}


\newcommand{\link}[1]{\href{../../../../Exercises/Apostol/Chapter_I_3.html\##1}{#1}}

\begin{document}

\section*{Theorem I.27}%
\label{sec:theorem-i.27}

Every nonempty set $S$ that is bounded below has a greatest lower bound; that
is, there is a real number $L$ such that $L = \inf{S}$.

\begin{proof}

  \link{Real.exists\_isGLB}

\end{proof}

\section*{Theorem I.29}%
\label{sec:theorem-i.29}

For every real $x$ there exists a positive integer $n$ such that $n > x$.

\begin{proof}

  \link{Real.exists\_pnat\_geq\_self}

\end{proof}

\section*{Theorem I.30 (Archimedean Property of the Reals)}%
\label{sec:theorem-i.30}

If $x > 0$ and if $y$ is an arbitrary real number, there exists a positive
integer $n$ such that $nx > y$.

\begin{proof}

  \link{Real.exists\_pnat\_mul\_self\_geq\_of\_pos}

\end{proof}

\section*{Theorem I.31}%
\label{sec:theorem-i.31}

If three real numbers $a$, $x$, and $y$ satisfy the inequalities
$$a \leq x \leq a + \frac{y}{n}$$
for every integer $n \geq 1$, then $x = a$.

\begin{proof}

  \link{Real.forall\_pnat\_leq\_self\_leq\_frac\_imp\_eq}

\end{proof}

\section*{Theorem I.32}%
\label{sec:theorem-i.32}

Let $h$ be a given positive number and let $S$ be a set of real numbers.
\begin{enumerate}[(a)]
  \item If $S$ has a supremum, then for some $x$ in $S$ we have
    $$x > \sup{S} - h.$$
  \item If $S$ has an infimum, then for some $x$ in $S$ we have
    $$x < \inf{S} + h.$$
\end{enumerate}

\begin{proof}

  \begin{enumerate}[(a)]
    \item \link{Real.sup\_imp\_exists\_gt\_sup\_sub\_delta}
    \item \link{Real.inf\_imp\_exists\_lt\_inf\_add\_delta}
  \end{enumerate}

\end{proof}

\section*{Theorem I.33 (Additive Property)}%
\label{sec:theorem-i.33}

Given nonempty subsets $A$ and $B$ of $\mathbb{R}$, let $C$ denote the set
$$C = \{a + b : a \in A, b \in B\}.$$

\begin{enumerate}[(a)]
  \item If each of $A$ and $B$ has a supremum, then $C$ has a supremum, and
    $$\sup{C} = \sup{A} + \sup{B}.$$
  \item If each of $A$ and $B$ has an infimum, then $C$ has an infimum, and
    $$\inf{C} = \inf{A} + \inf{B}.$$
\end{enumerate}

\begin{proof}

  \begin{enumerate}[(a)]
    \item \link{Real.sup\_minkowski\_sum\_eq\_sup\_add\_sup}
    \item \link{Real.inf\_minkowski\_sum\_eq\_inf\_add\_inf}
  \end{enumerate}

\end{proof}

\section*{Theorem I.34}%
\label{sec:theorem-i.34}

Given two nonempty subsets $S$ and $T$ of $\mathbb{R}$ such that
$$s \leq t$$
for every $s$ in $S$ and every $t$ in $T$. Then $S$ has a supremum, and $T$
has an infimum, and they satisfy the inequality
$$\sup{S} \leq \inf{T}.$$

\begin{proof}

  \link{Real.forall\_mem\_le\_forall\_mem\_imp\_sup\_le\_inf}

\end{proof}

\end{document}
