\documentclass{article}
\usepackage{amsmath}

\usepackage{amsfonts, amsthm}
\usepackage{hyperref}

\newtheorem{theorem}{Theorem}
\newtheorem{xtheoreminner}{Theorem}
\newenvironment{xtheorem}[1]{%
  \renewcommand\thextheoreminner{#1}%
  \xtheoreminner
}{\endxtheoreminner}

\hypersetup{colorlinks=true, urlcolor=blue}

\newcommand{\ceil}[1]{\left\lceil#1\right\rceil}
\newcommand{\floor}[1]{\left\lfloor#1\right\rfloor}
% The first argument refers to a relative path upward from a current file to
% the root of the workspace (i.e. where this `preamble.tex` file is located).
\newcommand{\lean}[4]{\href{#1/#2.html\##3}{#4}}


\newcommand{\larea}[2]{\lean{../..}{Bookshelf/Real/Geometry/Area}{#1}{#2}}
\newcommand{\lrect}[2]{\lean{../..}{Bookshelf/Real/Geometry/Rectangle}{#1}{#2}}

\begin{document}

The properties of area in this set of exercises are to be deduced from the
axioms for area stated in the foregoing section.

\section{Exercise 1}%
\label{sec:exercise-1}

Prove that each of the following sets is measurable and has zero area:

\subsection{Exercise 1a}%
\label{sub:exercise-1a}

A set consisting of a single point.

\begin{proof}

  Let $S$ be a set consisting of a single point.
  By definition of a \lrect{Real.Point}{Point}, $S$ is a rectangle in which all
    vertices coincide.
  By \larea{Choice-of-Scale}{Choice of Scale}, $S$ is measurable with area its
    width times its height.
  The width and height of $S$ is trivially zero.
  Therefore $a(S) = (0)(0) = 0$.

\end{proof}

\subsection{Exercise 1b}%
\label{sub:exercise-1b}

A set consisting of a finite number of points in a plane.

\begin{proof}

  We show for all $k > 0$, a set consisting of $k$ points in a plane is
  measurable with area $0$.

  \paragraph{Base Case}%

    Consider a set $S$ consisting of a single point in a plane.
    By \eqref{sub:exercise-1a}, $S$ is measurable with area $0$.

  \paragraph{Induction Step}%

    Define our induction hypothesis as, "Let $k > 0$ and assume a set consisting
      of $k$ points in a plane is measurable with area $0$."

    Consider a set $S_{k+1}$ consisting of $k + 1$ points in a plane.
    Pick an arbitrary point of $S_{k+1}$.
    Denote the set containing just this point as $T$.
    Denote the remaining set of points as $S_k$.
    By construction, $S_{k+1} = S_k \cup T$.
    By the induction hypothesis, $S_k$ is measurable with area $0$.
    By \eqref{sub:exercise-1a}, $T$ is measurable with area $0$.
    By the \larea{Additive-Property}{Additive Property}, $S_k \cup T$ is
      measurable, $S_k \cap T$ is measurable, and
      \begin{align}
        a(S_{k+1})
          & = a(S_k \cup T) \nonumber \\
          & = a(S_k) + a(T) - a(S_k \cap T) \nonumber \\
          & = 0 + 0 - a(S_k \cap T). \label{sub:exercise-1b-eq1}
      \end{align}

    \noindent
    There are two cases to consider:

    \subparagraph{Case 1}%

      $S_k \cap T = \emptyset$.
      Then it trivially follows that $a(S_k \cap T) = 0$.

    \subparagraph{Case 2}%

      $S_k \cap T \neq \emptyset$.
      Since $T$ consists of a single point, $S_k \cap T = T$.
      By \eqref{sub:exercise-1a}, $a(S_k \cap T) = a(T) = 0$.

    \vspace{8pt}
    \noindent
    In both cases, \eqref{sub:exercise-1b-eq1} evaluates to $0$, implying
      $a(S_{k+1}) = 0$ as expected.

  \paragraph{Conclusion}%

    By mathematical induction, it follows for all $n > 0$, a set consisting of
    $n$ points in a plane is measurable with area $0$.

\end{proof}

\subsection{Exercise 1c}%
\label{sub:exercise-1c}

The union of a finite collection of line segments in a plane.

\begin{proof}

  We show for all $k > 0$, a set consisting of $k$ line segments in a plane is
  measurable with area $0$.

  \paragraph{Base Case}%

    Consider a set $S$ consisting of a single line segment in a plane.
    By definition of a \lrect{Real.LineSemgnet}{Line Segment}, $S$ is a
      rectangle in which one side has dimension $0$.
    By \larea{Choice-of-Scale}{Choice of Scale}, $S$ is measurable with area its
      width $w$ times its height $h$.
    Therefore $a(S) = wh = 0$.

  \paragraph{Induction Step}%

    Define our induction hypothesis as, "Let $k > 0$ and assume a set consisting
      of $k$ line segments in a plane is measurable with area $0$."

    Consider a set $S_{k+1}$ consisting of $k + 1$ line segments in a plane.
    Pick an arbitrary line segment of $S_{k+1}$.
    Denote the set containing just this line segment as $T$.
    Denote the remaining set of line segments as $S_k$.
    By construction, $S_{k+1} = S_k \cup T$.
    By the induction hypothesis, $S_k$ is measurable with area $0$.
    By the base case, $T$ is measurable with area $0$.
    By the \larea{Additive-Property}{Additive Property}, $S_k \cup T$ is
      measurable, $S_k \cap T$ is measurable, and
      \begin{align}
        a(S_{k+1})
          & = a(S_k \cup T) \nonumber \\
          & = a(S_k) + a(T) - a(S_k \cap T) \nonumber \\
          & = 0 + 0 - a(S_k \cap T). \label{sub:exercise-1c-eq1}
      \end{align}

    \noindent
    There are two cases to consider:

    \subparagraph{Case 1}%

      $S_k \cap T = \emptyset$.
      Then it trivially follows that $a(S_k \cap T) = 0$.

    \subparagraph{Case 2}%

      $S_k \cap T \neq \emptyset$.
      Since $T$ consists of a single point, $S_k \cap T = T$.
      By the base case, $a(S_k \cap T) = a(T) = 0$.

    \vspace{8pt}
    \noindent
    In both cases, \eqref{sub:exercise-1c-eq1} evaluates to $0$, implying
      $a(S_{k+1}) = 0$ as expected.

  \paragraph{Conclusion}%

    By mathematical induction, it follows for all $n > 0$, a set consisting of
    $n$ line segments in a plane is measurable with area $0$.

\end{proof}

\section{Exercise 2}%
\label{sec:exercise-2}

Every right triangular region is measurable because it can be obtained as the
intersection of two rectangles. Prove that every triangular region is measurable
and that its area is one half the product of its base and altitude.

\begin{proof}

  TODO

\end{proof}

\end{document}
