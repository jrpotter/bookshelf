\documentclass{article}
\usepackage[shortlabels]{enumitem}

\usepackage{amsfonts, amsmath, amssymb, amsthm}
\usepackage{bigfoot}
\usepackage{comment}
\usepackage[shortlabels]{enumitem}
\usepackage{etoolbox}
\usepackage{environ}
\usepackage{fontawesome5}
\usepackage{mathabx, mathrsfs}
\usepackage{soul}
\usepackage{stmaryrd}
% Must load `xcolor` before `tcolorbox` and `tikz`.
\usepackage[dvipsnames]{xcolor}
\usepackage{tcolorbox}
\usepackage{tikz}
% `hyperref` comes after `xr-hyper`.
\usepackage{xr-hyper}
\usepackage{hyperref}

% Open "private" namespace.
\makeatletter

% ========================================
% General
% ========================================

\newcommand{\header}[2]{\title{#1}\author{#2}\date{}\maketitle}

% ========================================
% Dividers
% ========================================

\newcommand\@linespace{\vspace{10pt}}
\newcommand\linedivider{\@linespace\hrule\@linespace}
\WithSuffix\newcommand\linedivider*{\@linespace\hrule}
\newcommand\suitdivider{$$\spadesuit\;\spadesuit\;\spadesuit$$}

% ========================================
% Linking
% ========================================

\hypersetup{colorlinks=true, linkcolor=blue, urlcolor=blue}
\newcommand{\textref}[1]{\text{\nameref{#1}}}
\newcommand{\hyperlabel}[1]{%
  \label{#1}%
  \hypertarget{#1}{}}

% Links to theorems/statements/etc. that can be found in Mathlib4's index.
\newcommand\@leanlink[3]{%
  \textcolor{BlueViolet}{\raisebox{-4.5pt}{%
    \tikz{\draw (0, 0) node[yscale=-1,xscale=1] {\faFont};}}{-\;}}%
  \href{https://leanprover-community.github.io/mathlib4_docs/#1.html\##2}%
  {\color{BlueViolet}{#3}}}

\newcommand\lean[2]{%
  \noindent\@leanlink{#1}{#2}{#2}}
\WithSuffix\newcommand\lean*[2]{%
  \vspace{6pt}\lean{#1}{#2}}

\newcommand\leanp[3]{%
  \noindent\@leanlink{#1}{#2}{#3}}
\WithSuffix\newcommand\leanp*[3]{%
  \vspace{6pt}\leanp{#1}{#2}{#3}}

% Links to theorems/statements/etc. found in custom index.
\newcommand\@codelink[4]{%
  \textcolor{MidnightBlue}{\raisebox{-4.5pt}{%
    \tikz{\draw (0, 0) node[xshift=8pt] {\faCodeBranch};}}{-\;}}%
  \href{#1/#2.html\##3}%
  {\color{MidnightBlue}{#4}}}

\newcommand\coderef[3]{%
  \@codelink{#1}{#2}{#3}{#3}}
\newcommand\codepref[4]{%
  \@codelink{#1}{#2}{#3}{#4}}

% Macro to build our `code` commands relative to a given directory. For
% instance, we expect to have invocation `\makecode{..}` if the TeX file exists
% one directory deep from the root of our project..
\newcommand\makecode[1]{%
  \newcommand\code[2]{%
    \noindent\coderef{#1}{##1}{##2}}
  \WithSuffix\newcommand\code*[2]{%
    \vspace{6pt}\noindent\coderef{#1}{##1}{##2}}

  \newcommand\codep[3]{%
    \noindent\codepref{#1}{##1}{##2}{##3}}
  \WithSuffix\newcommand\codep*[3]{%
    \vspace{6pt}\noindent\codepref{#1}{##1}{##2}{##3}}
}

% ========================================
% Admonitions
% ========================================

\NewEnviron{note}{%
  \begin{tcolorbox}[%
      sharp corners,
      fonttitle=\sffamily\bfseries,
      toptitle=2pt,
      bottomtitle=2pt,
      coltitle=black!80!white,
      colback=yellow!30,
      colframe=yellow!80!black,
      title=Note]
    \BODY
  \end{tcolorbox}}

% ========================================
% Statements
% ========================================

\newcommand\@statement[1]{%
  \linedivider*\paragraph{\normalfont\normalsize\textit{#1.}}}
\newenvironment{answer}{\@statement{Answer}}{\hfill$\square$}
\renewenvironment{proof}{\@statement{Proof}}{\hfill$\square$}

\newtheorem{corollaryinner}{Corollary}
\newenvironment{corollary}[1][]{%
  \ifstrempty{#1}
    {\corollaryinner}
    {\renewcommand\thecorollaryinner{#1}\corollaryinner}
}{\endcorollaryinner}

\newtheorem{lemmainner}{Lemma}
\newenvironment{lemma}[1][]{%
  \ifstrempty{#1}
    {\lemmainner}
    {\renewcommand\thelemmainner{#1}\lemmainner}
}{\endlemmainner}

\newtheorem{theoreminner}{Theorem}
\newenvironment{theorem}[1][]{%
  \ifstrempty{#1}
    {\theoreminner}
    {\renewcommand\thetheoreminner{#1}\theoreminner}
}{\endtheoreminner}

% ========================================
% Status
% ========================================

\DeclareRobustCommand{\defined}[1]{%
  \texorpdfstring{\color{darkgray}\faParagraph\ #1}{#1}}
\DeclareRobustCommand{\verified}[1]{%
  \texorpdfstring{\color{teal}\faCheckCircle\ #1}{#1}}
\DeclareRobustCommand{\unverified}[1]{%
  \texorpdfstring{\color{olive}\faCheckCircle[regular]\ #1}{#1}}
\DeclareRobustCommand{\pending}[1]{%
  \texorpdfstring{\color{Fuchsia}\faPencil*\ #1}{#1}}
\DeclareRobustCommand{\sorry}[1]{%
  \texorpdfstring{\color{Maroon}\faExclamationCircle\ #1}{#1}}

% ========================================
% Math
% ========================================

\newcommand{\abs}[1]{\left|#1\right|}
\newcommand{\ceil}[1]{\left\lceil#1\right\rceil}
\newcommand{\dom}[1]{\textop{dom}{#1}}
\newcommand{\fld}[1]{\textop{fld}{#1}}
\newcommand{\floor}[1]{\left\lfloor#1\right\rfloor}
\newcommand{\icc}[2]{\left[#1, #2\right]}
\newcommand{\ico}[2]{\left[#1, #2\right)}
\newcommand{\img}[2]{#1\!\left\llbracket#2\right\rrbracket}
\newcommand{\ioc}[2]{\left(#1, #2\right]}
\newcommand{\ioo}[2]{\left(#1, #2\right)}
\newcommand{\powerset}[1]{\mathscr{P}#1}
\newcommand{\ran}[1]{\textop{ran}{#1}}
\newcommand{\textop}[1]{\mathop{\text{#1}}}
\newcommand{\ubar}[1]{\text{\b{$#1$}}}

\let\oldemptyset\emptyset
\let\emptyset\varnothing

% Close off "private" namespace.
\makeatother


\newcommand{\link}[1]{\href{/../../../../OneVariableCalculus/Apostol/Chapter_I_3.html\##1}{#1}}

\begin{document}

\section*{Theorem I.27}%
\label{sec:theorem-i.27}

Every nonempty set $S$ that is bounded below has a greatest lower bound; that
is, there is a real number $L$ such that $L = \inf{S}$.

\begin{proof}

  \link{Real.exists\_isGLB}

\end{proof}

\section*{Theorem I.29}%
\label{sec:theorem-i.29}

For every real $x$ there exists a positive integer $n$ such that $n > x$.

\begin{proof}

  \link{Real.exists\_pnat\_geq\_self}

\end{proof}

\section*{Theorem I.30 (Archimedean Property of the Reals)}%
\label{sec:theorem-i.30}

If $x > 0$ and if $y$ is an arbitrary real number, there exists a positive
integer $n$ such that $nx > y$.

\begin{proof}

  \link{Real.exists\_pnat\_mul\_self\_geq\_of\_pos}

\end{proof}

\section*{Theorem I.31}%
\label{sec:theorem-i.31}

If three real numbers $a$, $x$, and $y$ satisfy the inequalities
$$a \leq x \leq a + \frac{y}{n}$$
for every integer $n \geq 1$, then $x = a$.

\begin{proof}

  \link{Real.forall\_pnat\_leq\_self\_leq\_frac\_imp\_eq}

\end{proof}

\section*{Theorem I.32}%
\label{sec:theorem-i.32}

Let $h$ be a given positive number and let $S$ be a set of real numbers.
\begin{enumerate}[(a)]
  \item If $S$ has a supremum, then for some $x$ in $S$ we have
    $$x > \sup{S} - h.$$
  \item If $S$ has an infimum, then for some $x$ in $S$ we have
    $$x < \inf{S} + h.$$
\end{enumerate}

\begin{proof}

  \begin{enumerate}[(a)]
    \item \link{Real.sup\_imp\_exists\_gt\_sup\_sub\_delta}
    \item \link{Real.inf\_imp\_exists\_lt\_inf\_add\_delta}
  \end{enumerate}

\end{proof}

\section*{Theorem I.33 (Additive Property)}%
\label{sec:theorem-i.33}

Given nonempty subsets $A$ and $B$ of $\mathbb{R}$, let $C$ denote the set
$$C = \{a + b : a \in A, b \in B\}.$$

\begin{enumerate}[(a)]
  \item If each of $A$ and $B$ has a supremum, then $C$ has a supremum, and
    $$\sup{C} = \sup{A} + \sup{B}.$$
  \item If each of $A$ and $B$ has an infimum, then $C$ has an infimum, and
    $$\inf{C} = \inf{A} + \inf{B}.$$
\end{enumerate}

\begin{proof}

  \begin{enumerate}[(a)]
    \item \link{Real.sup\_minkowski\_sum\_eq\_sup\_add\_sup}
    \item \link{Real.inf\_minkowski\_sum\_eq\_inf\_add\_inf}
  \end{enumerate}

\end{proof}

\section*{Theorem I.34}%
\label{sec:theorem-i.34}

Given two nonempty subsets $S$ and $T$ of $\mathbb{R}$ such that
$$s \leq t$$
for every $s$ in $S$ and every $t$ in $T$. Then $S$ has a supremum, and $T$
has an infimum, and they satisfy the inequality
$$\sup{S} \leq \inf{T}.$$

\begin{proof}

  \link{Real.forall\_mem\_le\_forall\_mem\_imp\_sup\_le\_inf}

\end{proof}

\end{document}
